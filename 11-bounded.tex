\chapter{Bounded Rationality}

% Would be nice to fit the unmarked clock puzzle in here somehow.

\section{Models and reality}

We have studied an abstract model of rational agents. The model assumes that an
agent has some idea of what the world might be like, which we represent by a
credence function $\Cr$ over a suitable space of propositions. The agent also
has some goals or values or desires, represented by a (possibly partial) utility
function $\U$ on the same space of propositions. The credence function is
assumed to satisfy the formal rules of the probability calculus. It evolves over
time by conditionalizing on sensory information, and it satisfies some further
constraints like the Probability Coordination Principle. An agent's utility
function is assumed to satisfy Jeffrey's axiom, so that it is jointly determined
by the agent's credences and their ``intrinsic utility function'' that assigns a
value to the agent's ``concerns'' -- combinations of things the agent ultimately
cares about. These intrinsic utilities may in turn be determined by aggregating
subvalues. When the agent faces a choice, they are assumed to choose an act that
maximizes the credence-weighted average of the utility of the possible outcomes.

Our model is really a family of models, as there are different ways of filling
in the details. Should expected utility be understood causally or evidentially?
Should credences satisfy some version of the Indifference Principle? Should we
rule out some basic desires as irrational? Should we require time consistency?
Should we impose constraints on how basic desires may change over time?
Different answers yield different models.

Each model in this family can be understood either \textbf{normatively} or
\textbf{descriptively}. Understood normatively, the model would purport to
describe an ideal to which real agents should perhaps aspire. Understood
descriptively, the model would purport to describe the attitudes and choices of
ordinary humans.

It is a commonplace in current economics and psychology that our model is
descriptively inadequate (no matter how the details are spelled out): that real
people are not expected utility maximizers. In itself, this is not necessarily a
problem -- not even for the descriptive interpretation of our models. Remember
that ``all models are wrong''. With the possible exception of the standard model
of particle physics, the purpose of a model is to identify interesting and
robust patterns in the phenomena, not to get every detail right. Nonetheless, it
is worth looking at how our model aligns with reality, and what we
could change to make it more realistic.

Many supposed cases where people are said to violate the MEU Principle are not
counterexamples to the descriptive adequacy of the model we have been studying.
Our model can easily accommodate agents who care about risk or fairness or
regret (chapter \ref{ch:risk}). We can accommodate altruistic behaviour (section
\ref{sec:decision-matrices}), the endowment effect (section
\ref{sec:sources-utility}), and apparent failures of time consistency (section
\ref{sec:separability-time}).

Other phenomena are harder to accommodate. People often make mistakes when
evaluating the impact of inconclusive information. They don't take into account
the ``base rate'' (section \ref{sec:conditionalization}) or the fact that the
information comes from a biased source. They ignore evidence that goes against
their opinions.

More simply, most people are bad at maths. Suppose I offer you £100 for telling
me the prime factors of 82,717. You have 10 seconds. All you'd have to do, to
get the money, is utter `181 and 457'. Moreover, that this is the correct answer
logically follows from simpler facts of which you are highly confident. By the
rules of probability, you should be confident that `181 and 457' is the correct
answer. But you are not.

\begin{exercise}{3}
  Explain why, if some proposition $C$ is entailed by two propositions $A$ and
  $B$ whose probability is greater than 0.99, then the probability of $C$ is
  greater than 0.98.
\end{exercise}

In 1913, Ernst Zermelo proved that in the game of chess, there is either a
strategy for the starting player, White, that guarantees victory no matter what
Black does, or there is such a strategy for Black, or there is a strategy for
either player to force a draw. Consequently, if two ideal Bayesian agents sat
down to a game of chess, and their only interest was in winning, they would
either agree to a draw or one of them would resign immediately, before the first
move. Real people don't play like this.

Another respect in which real people plausibly deviate from our model is that
they often overlook certain options. You go to the shop, but forget to buy soap.
You walk along the highway because it doesn't occur to you that you could take
the nicer route through the park. The relevant options (buying soap, taking the
nicer route) are available to you, and they are better by the lights of your
beliefs and desires, so it is a mistake that you don't choose them.

Relatedly, real people are forgetful. I don't remember what I had for dinner
last Monday. As an ideal Bayesian agent, I would still know what I had for
dinner on every day of my life.

\begin{exercise}{2}
  Show that if an agent conditionalizes on some information $E$ then their
  credence in $E$ will remain at 1 as long as the agent only changes their
  beliefs by further applications of conditionalisation. (Conditionalization was
  introduced in section \ref{sec:conditionalization}.)
\end{exercise}

\cmnt{%
  Other issues are less clear. E.g., it is widely claimed that real agents are
  not ``logically omniscient'', meaning that they are not certain of every
  logical truth, in violation of Kolmogorov's axiom 2. But it is not obvious
  what it would mean for an agent to have credence less than 1 in a logical
  truth, and therefore it is also not obvious whether people actually have such
  attitudes. Indeed, if `credence' is a technical term defined by our model,
  then so far it is trivially true that everyone is logically omniscient. We
  would need an alternative model.%
} %

There is also indirect evidence that our model does not fit real agents in every
respect. The evidence comes from research on artificial intelligence, where our
model forms the background for much recent research. Various parts of the model
-- including the MEU Principle and the Principle of Conditionalization -- turn
out to be computationally intractable. Real agents with limited cognitive
resources, it seems, couldn't possibly conform to our model.

\section{Avoiding computational costs}

Before we look at ways of making our model more realistic, I want to address
another common misunderstanding.

Suppose you walk back to the shop to buy soap. At any point on your way, you
could change course. You could decide to turn around, or start running. You
could check if your shoe laces are tied. You could mentally compute $181 + 457$,
or start humming the national anthem. There are millions of things you could do.
Many of these would lead to significantly different outcomes, especially if you
consider long-term consequences. (Hitler almost certainly would not have existed
if hours or even months before his conception, his father had decided to run
rather than walk to buy soap.) Some authors take the MEU Principle to imply that
at each point on your walk, you should explicitly consider all your options,
envisage all their possible outcomes, assess their utility and probability, and
on that basis compute their expected utility. This is clearly unrealistic and
infeasible.

But the MEU Principle requires no such thing. The MEU Principle says that
rational agents choose acts that maximize expected utility; it specifies
\emph{which acts} an agent should choose, given their beliefs and desires. It says
nothing about the internal processes that lead to these choices. It does not say
that the agent must explicitly consider all their options and compute expected
utilities.

\begin{exercise}{2}
  The opposite is closer to the truth. Suppose an agent has a choice between
  turning left ($L$), turning right ($R$), and sitting down to compute the
  expected utility of $L$ and $R$ and then choosing whichever comes out best.
  Let $C$ be this third option. If computing expected utilities involves some
  costs in terms of effort or time, then either $L$ or $R$ generally has greater
  expected utility than $C$. Explain why.
\end{exercise}

The MEU Principle does not require calculating expected utilities. But this
raises a puzzle. An agent who conforms to our model always chooses acts with
greatest expected utility. How are they supposed to do this without calculating?
It doesn't seem rational to choose one's acts randomly and maximize expected
utility by sheer luck.

Part of the answer is that our model abstracts away from cognitive limitations.
Agents who conform to our model have no need to calculate anything. If their
evidence entails that a certain act maximizes expected utility, then they are
already certain that the act maximizes expected utility: anything that is
entailed by their evidence automatically has credence 1.

The idea that expected utility maximizers would constantly have to go through
intricate computations also assumes that credences and utilities are
conceptually prior to choices. On a preference-based approach, preferences and
choices come first. The MEU Principle boils down to certain constraints on
preferences, which in turn boil down to constraints on choices. One might hope
that even real people, who aren't logically omniscient, can reliably satisfy
these constraints without computing expected utilities.

% In many circumstances, simple alternatives to computing expected utilities
% reliably lead to optimal choices. As the psychologist Gerd Gigerenzer once
% pointed out, if you want to catch a flying ball, an efficient alternative to
% computing the ball's trajectory -- which is generally intractable -- is to move
% around in such a way that the angle between you and the ball remains within a
% certain range. This ensures that you'll eventually stand where the ball will
% arrive.

% If you desire to catch the ball,
% following Gigerenzer's heuristic will maximize expected utility. You don't need
% to consciously compute anything, and you don't need to conceptualize what you're
% doing as maximizing expected utility.

\cmnt{%
  Another part of the answer is that ideal Bayesian agents have no need to
  compute anything: if they are certain of some facts, and these facts entail
  other facts, then the probability axioms entail that they are also certain of
  the other facts. For ideal Bayesian agents, computations are a costly process
  of arriving at results that are already known.%
}%

\begin{exercise}{1}
  Suppose you're a musician in the middle of a performance. Trying to compute
  the expected utility of all the notes you could play next would probably
  derail your play. Even if it wouldn't, it would change your experience of
  playing, probably for the worse. Give another example where conceptualizing
  one's acts as maximizing expected utility would undermine the value of
  performing the acts.
\end{exercise}

In many decision situations, there is no need for sophisticated calculations
because one of the acts clearly dominates the others. Whether this is the case
depends on the agent's utility function. This suggests that wemight reduce the
computational costs of decision-making by tweaking our utilities.

For example, suppose you assign significant (sub)value to obeying orders. Doing
whatever you're ordered to do is then a reliable way of maximizing expected
utility, and it requires little cognitive effort. Similarly if you value
imitating whatever your peers are doing.

\cmnt{%
  Endowing individuals with a desire for social conformity might be an
  evolutionary trick to ensure that people make reasonably simple choices
  without having to reason too much. (Conformity has other evolutionary
  advantages as well.)%
} %

Our capacity for planning and commitment can also be seen in this light. Before
you went to the shop, you probably decided to go to the shop. The direct result
of your decision was an intention to go to the shop.%
%
\cmnt{%
  Somewhat confusingly, according to our model, this was not a
  decision to go to the shop. That's because, at the time, going to
  the shop was not strictly speaking an available act, since was
  partly outside your control. Someone could have prevented you from
  leaving the house. You could have gotten into an accident while
  crossing the street. Your future self could have been seduced by the
  sight of the pub and stopped for a few beers. When you decided to go
  to the shop, the ``act'' you chose was an act of commitment or
  planning. You formed the intention of going to the shop.%
} %
%
Once an intention or plan is formed, we are motivated to execute it. Revising a
plan or overturning a commitment has negative (sub)value. Consequently, once
you've formed an intention, simply following it reliably maximizes expected
utility. You don't need to think any more about what to do unless you receive
surprising new information or your basic values suddenly change. (This is true
even if you've made a mistake when you originally formed the intention.)

Habits can play a similar role. Most of us spend little effort deciding whether
we should brush our teeth in the morning. We do it out of habit. Habitual
behaviour is computationally cheap, and it can reliably maximize expected
utility -- especially if we assign (sub)value to habitual behaviour. And we do,
at least on a motivational conception of desire: habits motivate.

The upshot is that various cognitive strategies that are often described as
alternatives to computing expected utilities -- habits, instincts, heuristics,
etc.\ -- may well be efficient techniques for maximizing expected utility. Far
from ruling out such strategies, our model predicts that we should use them.

\cmnt{%
  Still, one might complain that our model lacks detail.  We could
  model pointless habitual behaviour as simply MEU, but to understand
  why someone might have the relevant utilities, it is useful to
  develop a more complex model that distinguishes between deliberate
  and habitual behaviour. (Such ``two systems'' models are common both
  in psychology and neuroscience; see e.g.\ Dickinson A. Actions and
  habits: the development of behavioural autonomy. Philosophical
  Transactions of the Royal Society B: Biological Sciences. 1985;
  308:67–78 and 2 system theory.)
} %

An example in which something like this might play a role is Ellsberg's Paradox,
another classical ``counterexample'' to the MEU Principle.

\begin{example}(Ellsberg's Paradox)
  An urn contains 300 balls. 100 of the balls are red, the others are
  green or blue, in unknown proportion. A ball is drawn at random from
  the urn. Which of the following two gambles ($A$ and $B$) do you prefer?
  % 
  \begin{center}
  \begin{tabular}{|r|c|c|c|}\hline
    \gr & \gr Red & \gr Green & \gr Blue \\\hline
    \gr $A$ & £1000 & £0 & £0 \\\hline
    \gr $B$ & £0 & £1000 & £0  \\\hline
  \end{tabular}
  \end{center}
  %
  Next, which of $C$ and $D$ do you prefer?
  % 
  \begin{center}
  \begin{tabular}{|r|c|c|c|}\hline
    \gr & \gr Red & \gr Green & \gr Blue \\\hline
    \gr $C$ & £1000 & £0 & £1000 \\\hline
    \gr $D$ & £0 & £1000 & £1000 \\\hline
  \end{tabular}
  \end{center}
\end{example}
%
Many people prefer $A$ to $B$ and $D$ to $C$. Like in Allais's Paradox, there is
no way of assigning utilities to the monetary outcomes that supports these
preferences.

\begin{exercise}{1}
  Assume the outcomes in Ellsberg's paradox are described correctly and you
  prefer more money to less. By the Probability Coordination Principle,
  $\Cr(\text{Red}) = \nicefrac{1}{3}$. What would your credences in Green
  and Blue have to be so that $\EU(A) > \EU(B)$? What would they
  have to be so that $\EU(D) > \EU(C)$?
\end{exercise}

In Ellsberg's Paradox, risk aversion doesn't seem to be at issue. What makes the
difference is that you know the objective probability of winning for options $A$
and $D$: it is $\nicefrac{1}{3}$ for $A$ and $\nicefrac{2}{3}$ for $D$. You
don't know the objective probability of winning with $B$ and $C$, since you have
too little information about the non-red balls.

Why does this matter? One explanation is that people simply prefer lotteries, in
which the outcomes have known objective probabilities, to gambles in which the
outcomes can only be assigned subjective probabilities. With such a utility
function, the outcome labelled `£1000' in $A$ is actually better than the
corresponding outcome in $C$, because only the former involves having chosen a
lottery.

% \begin{exercise}{1}
%   The explanation of the Ellsberg preferences that I just outlined
%   makes the preferences conform to the MEU Principle by redescribing
%   the outcomes. Is the redescription global or local in the sense of
%   chapter \ref{ch:risk}? 
% \end{exercise}

But why would agents prefer lotteries? A possible answer is that such a
preference tends to reduce computational costs. If you know the objective
probabilities of a state, it is easy to figure out the credence you should give
to the state: it should match the objective probabilities. If you don't know the
objective probability, more work may be required to figure out the extent to
which the state is supported by your total evidence. In Ellsberg's Paradox,
$\Cr(\text{Red})$ is a easier to figure out than $\Cr(\text{Green})$ and
$\Cr(\text{Blue})$. If you have a preference for lotteries, you don't need to
figure out $\Cr(\text{Green})$ and $\Cr(\text{Blue})$: from eyeballing the
options, you can already see that the expected monetary payoff of $A$ and $B$ is
approximately the same (as is the expected payoff of $C$ and $D$); a preference
for lotteries tips the balance in favour of $A$ (and $D$).

% We still haven't fully resolved our puzzle. Should real agents follow the MEU
% Principle, in hard cases, where no heuristic would help? If we say yes, then
% we either say that should enter into lengthy computations, or that they should
% follow the Principle by luck. Both seem bad. (Although one could argue that
% normal people care about being in rational control of their choices, or that
% they should: we don't want to make important choices randomly, but based on
% good reasons. I.e., we actually put higher utility to the best choice
% resulting from computation.) So the answer is probably 'no'. 

\section{Reducing computational costs}\label{sec:ai}

I will now review a few ideas from theoretical computer science for rendering
our models computationally tractable.

Imagine we want to design a robot -- an artificial agent with a probabilistic
representation of its environment and some goals. Let's assume that we want our
agent to assign credences and utilities to a total of 50 logically independent
propositions $A_1,\ldots,A_{50}$ (an absurdly small number). How large of a
database do we need?

You might think that we need 50 records for the probabilities and 50 for the
utilities. But we generally can't compute $Cr(A \land B)$ or $\Cr(A \lor B)$
from $\Cr(A)$ and $\Cr(B)$. Nor can we compute $\U(A \land B)$ or $\U(A \lor B)$
from $\U(A)$ and $\U(B)$. If we want to determine the agent's entire credence
and utility functions (without further assumptions), we need to store at least
the probability and utility of every ``possible world'' -- every maximally
consistent conjunction of $A_1,\ldots,A_{50}$ and their negations.

\begin{exercise}{3}
  Explain why the probability of every proposition that can be defined in terms
  of $A_1,\ldots,A_{50}$ can be computed from the probability assigned to these
  ``worlds''. Then explain why the utility of every such proposition can be
  computed from the probability and utility assigned to the worlds.
\end{exercise}

There are $2^{50} = 1,125,899,906,842,624$ maximally consistent conjunctions of
$A_1,\ldots,A_{50}$ and their negations. Since we need to store both credences
and utilities, we need a database with $2,251,799,813,685,248$ records. (I am
exaggerating. Once we've fixed the probability of the first
1,125,899,906,842,623 worlds, the probability of the last world is 1 minus the
sum of the others, so we really only need $2,251,799,813,685,247$ records.)

We'll need to buy a lot of hard drives for our robot if we want to store 2
quadrillion floating point numbers. Worse, updating all these records in
response to sensory information, or computing expected utilities on their basis,
will take a very long time, and use a large amount of energy.

In chapters \ref{sec:basic-desire} and \ref{ch:separability}, we have
encountered two tricks that allow us to simplify the representation of an
agent's utility function. First, if the agent cares about some attributes of the
world and not about others, it is enough to store the agent's utility for her
``concerns'': the maximally consistent conjunctions of the attributes they care
about (section \ref{sec:basic-desire}). If, for example, our robot only cares
about the possible combinations of 20 among the 50 propositions
$A_1,\ldots,A_n$, we only need to store $2^{20}$ values. Second, if our robot's
preferences are separable with respect to these attributes, then the value of
any combination of the 20 propositions and their negations can be determined by
adding up relevant subvalues (section \ref{sec:additivity}). We can cut down
the number of utility records from $2^{20}$ to $2 \cdot 20 = 40$.

Similar tricks are available for the agent's credence function. Mirroring the
first trick, we could explicitly store only the robot's credence in certain sets
of worlds, and assume that its credence is distributed uniformly within these
sets. The trick can be extended to non-uniform distributions. For example,
suppose our robot has imperfect information about how far it is from the next
charging station. Instead of explicitly storing a probability for every possible
distance (1m, 2m, 3m, \ldots), we might assume that the robot's credence over
these possibilities follows a Gaussian distribution, which can be specified by
two numbers (mean and variance). Researchers in artificial intelligence make
heavy use of this trick.

An analogue of separability, for credences, is probabilistic independence. If
$A$ and $B$ are probabilistically independent, then
$\Cr(A\land B) = \Cr(A) \cdot \Cr(B)$. If all the 50 propositions
$A_1,\ldots,A_{50}$ are mutually independent, then we can fix the probability of
all possible worlds (and therefore of all logical combinations of the 50
propositions) by specifying their individual probability.

Independence is sometimes plausible. Whether the next charging station is 100
meters away plausibly doesn't depend on whether the outside temperature is above
20\celsius. For many other propositions, however, independence is implausible.
On the supposition that it is warm outside ($W$), it may well be more likely
that the window is open ($O$), or that there are people on the street ($P$),
than on the supposition that it isn't warm ($\neg W$). If our agent is unsure
whether it is warm, it follows that $\Cr(O/W) > \Cr(O)$, and
$\Cr(P/W) > \Cr(P)$. We can't assume probabilistic independence across all the
50 propositions $A_1,\ldots,A_{50}$.

Even where independence fails, however, we often have \textbf{conditional
  independence}. If warm temperatures make it more likely that the window is
open and that there are people on the street, then an open window is also
evidence that there are people on the street: $\Cr(P/O) > \Cr(P)$. So $P$ and $O$
are not independent. However, \emph{on the supposition that it is warm outside},
the window being open may no longer increase the probability of people on the
street:
\[
  \Cr(P/O \land W) = \Cr(P/ W).
\]
In this case, we say that $P$ and $O$ are independent \emph{conditional on} $W$.

Now consider the possible combinations of $W$, $P$, $O$ and their negations. By
the probability calculus (compare exercise \ref{e:chain-rule}),
\[
  \Cr(W \land P \land O) = \Cr(W) \cdot \Cr(O/W) \cdot \Cr(P/O \land W).
\]
By the above assumption of conditional independence, this simplifies to
\[
  \Cr(W \land P \land O) = \Cr(W) \cdot \Cr(O/W) \cdot \Cr(P/W).
\]

In general, with the assumption of conditional independence, we can fix the
probability of all combinations of $W$, $P$, $O$, and their negations by
specifying the probability of $W$, the probability of $P$ conditional on $W$ and
on $\neg W$, and the probability of $O$ conditional on $W$ and on $\neg W$. The
number of required records shrinks from $2^3-1 = 7$ to 5. This may not look all
that impressive, but the method really pays off if more than three propositions
are involved.

The present technique for exploiting conditional independence to simplify
probabilistic models is formalized in the theory of \textbf{Bayesian networks}
(or \textbf{Bayes nets}, for short). Bayes nets have proved useful in wide range
of applications.

A special case of Bayes nets%
\cmnt{%
  called \textbf{Dynamic Decision Networks}%
} %
is widely used in artificial intelligence to model decision-making agents.

A decision maker needs information not only about the present state of the
world, but also about the future. We can represent a history of states as a
sequence $S_1$, $S_2$, $S_3$,\ldots, where $S_1$ is a particular hypothesis about
the present state, $S_2$ about the next state, and so on. If there are 100
possible states at any given time, there will be
$100^{10} = 100,000,000,000,000,000,000$ possible histories with length 10.
Instead of storing individual probabilities for each of these possibilities, it
helps to assume that a later state (probabilistically) depends only on its
immediate predecessor, so that $\Cr(S_3\,/\,S_1 \!\land\! S_2) = \Cr(S_3/S_2)$. This is
known as the \textbf{Markov assumption}. It reduces the number of records we'd
need to store from $100^{10}$ to 990,100.%
\cmnt{%
  It also simplifies belief updates through conditionalization. Assuming that
  sensory information only carries direct information about the present state of
  the world, we only need to store $\Cr(E/S)$ for any evidence proposition $E$
  and state $S$.%
} %

To further simplify the task of decision-making, computer scientists usually
assume that the decision maker's intrinsic preferences are stationary and
separable across times, so that the value of a history of states is a discounted
sum of a subvalue for individual states. To specify the whole utility
function, we then only need to store the discounting factor $\delta$ and 100
values for the individual states. The task of conditionalization can also be
simplified, by assuming that sensory evidence only contains direct information
about the present state of the world.

These simplifications define what computer scientists call a `\textbf{POMDP}': a
\textbf{Partially Observable Markov Decision Process}. There is a simple
recursive algorithm for computing expected utilities in POMDPs.

\cmnt{%
  \begin{exercise}
    Explain one respect in which real decision situations are not
    adequately modelled by POMDPs.
  \end{exercise}
} %

In practice, even these simplifications generally don't suffice to make
conditionalization and expected utility maximization tractable. Further
simplifications are needed. It often helps to ignore states in the distant
future and let the agent maximize the expected total utility in the next few
states only. Several techniques have been developed that allow an efficient
\emph{approximate} computation of expected utilities and posterior
probabilities. These techniques are often supplemented by a meta-decision
process that lets the system choose a level of precision: when a lot is at
stake, it is worth spending more effort on getting the computations right.

While originating in theoretical computer science, these models and techniques
have in recent years had a great influence on our models of human cognition.
There is evidence that when our brain processes sensory information or decides
on a motor action, it employs the same techniques computer scientists have found
useful in approximating the Bayesian ideal. Several quirks of human perception
and decision-making have been argued to be a consequence of the shortcuts our
brain uses to approximate conditionalization and computing expected utilities.


\section{``Non-expected utility theories''}\label{sec:prospect-theory}

Meanwhile, researchers at the intersection of psychology and economics have also
tried to develop more realistic models of decision-making. The most influential
of these alternatives is \textbf{prospect theory}, developed by Daniel Kahneman
and Amon Tversky in the 1970s-1990s.

Prospect theory has to be understood on the background of a highly restricted
version of decision theory that dominates economics. The highly restricted
theory assumes that utility is only defined for money and other material goods,
and it only deals with choices between lotteries, where the objective
probabilities are known. People are assumed to want more money and goods, but
with declining marginal utility. When you find social scientists discuss
``Expected Utility Theory'', this highly restricted theory is what they usually
have in mind. Prospect theory now proposes four main changes.

1. \emph{Reference dependence}. According to prospect theory, agents classify
possible outcomes into gains and losses, by comparing the outcomes with a
contextually determined reference point. Outcomes better than the reference
point are modelled as having positive utility, outcomes worse than the reference
point have negative utility.

2. \emph{Diminishing sensitivity}. Prospect theory holds that both gains and
losses have diminishing marginal utility: the same objective difference in
wealth makes a larger difference in utility near the reference point than
further away, on either side. For example, the utility difference between a loss
of £100 and a loss of £200 is greater than that between a loss of £1000 and a
loss of £1100. This predicts that people are risk averse about gains but risk
seeking about losses: they prefer a sure gain of £500 to a 50 percent chance of
£1000, but they prefer a 50 percent chance of losing £1000 to losing £500 for
sure.

3. \emph{Loss aversion}: According to prospect theory, people are more sensitive
to losses than to gains of the same magnitude. The utility difference between a
loss of £100 and a loss of £200 is greater than that between a gain of £200 and
a gain of £100. This explains why many people turn down a lottery in which they
can either win £110 or lose £100, with equal probability.

4. \emph{Probability weighting}. According to prospect theory, the outcomes are
weighted not by their objective probability, but by transformed probabilities
known as `decision weights' that are meant to reflect how seriously people take
the relevant states in their choices. Decision weights generally overweight
low-probability outcomes. Thus probability 0 events have weight 0, probability 1
events have weight 1, but in between the weight curve is steep at the edges and
flatter in the middle: probability 0.01 events might have weight 0.05,
probability 0.02 events weight 0.08, \ldots, probability 0.99 events weight
0.95. Among other things, this is meant to explain why people play the lottery,
and why they tend to pay a high price for certainty: they prefer a settlement of
£90000 over a trial in which they have a 99\% chance of getting £100000 but a
1\% chance of getting nothing.%
\cmnt{%
  This ``certainty effect'' turns into a ``possibility effect'' at the other end
  of the scale.%

  What is supposedly captured by probability weighting is that an increase from
  a 0\% to a 1\% chance of loss significantly taints an option, much more than
  e.g.\ an increase from 20\% to 21\%. At the other end of the spectrum, the
  difference between 99\% and 100\% again matters a lot, much more than that
  between 79\% and 80\%. We could make the same predictions by assuming that
  people have preferences about risks, including global features in the utility
  function.%
} %
\cmnt{%
  Kahneman notes that buying a lottery ticket also buys ``the right to dream
  pleasantly of winning'', and that by buying insurance you also ``eliminate a
  worry and purchase a peace of mind'' \cite[318]{kahneman11thinking}, but he
  nevertheless resists broadening the utility function accordingly. %
} %

Prospect theory is clearly an alternative to the simplistic economical model
mentioned above. It is not so obvious whether it is an alternative to the more
liberal model that we have been studying. Diminishing sensitivity and loss
aversion certainly don't contradict our model. Reference dependence and
probability weighting are a little more subtle.

Our model assumes that if an agent knows the objective probability of a state,
then in decision-making she will weight that state in proportion to the known
probability. Prospect theory says that people don't actually do this. If we
measure an agent's credences in terms of preferences or choices, then the
decision weights of prospect theory are the agent's credences: they play
precisely the role of credences in guiding behaviour. From this perspective,
prospect theory assumes that people systematically violate the Probability
Coordination Principle. Their credence in low-probability events is greater than
the known objective probability.

Some have argued that the observations that motivate probability weighting are
better explained by redescribing the outcomes and allowing people to care about
things like risk or fairness. But there is evidence that people really do fail
to coordinate their beliefs with known objective probabilities, especially if
the probabilities are communicated verbally.  People's decision weights
tend to be closer to the objective probabilities if they have experienced the
probabilities as relative frequencies in repeated trials.%
\cmnt{%
  \cite[331]{kahneman11thinking}%
} %
% By contrast, when people reason explicitly about probabilities, systematic
% mistakes like the base rate fallacy are very common.

\cmnt{%
  The mere hypothesis that people's credences deviate from objective
  probabilities in a certain way however arguably doesn't get at the heart of
  the matter. What we'd need is a more general model explaining the difference
  between verbal and non-verbal presentation, and the particular mistakes people
  make.%
} %

\cmnt{%
  When you decide not to switch in Monty Hall, your credences are in line with
  your choices, but they violate other norms. Defining an alternative model of
  credence is not easy.%
} %

Reference dependence may also raise a genuine challenge. Many forms of reference
dependence can easily be accommodated in our model. We can allow that people
care about how much they own in comparison to what they have owned before, or in
comparison to what their peers own. But sometimes the reference point is
affected by intuitively irrelevant features of the context, and this is harder
to square with our model.

\begin{exercise}{1}
  When people compete in sports, average performance sometimes seems to function
  as a reference point, insofar as the effort people put in to avoid performing
  below average is higher than the effort they put in to exceed the average.%
  \cmnt{%
    \cite[303f.]{kahneman11thinking}.%
  } %
  Can you explain this observation by ``redescribing the outcomes'' in the model
  we have studied, without appealing to reference points?
\end{exercise}

\cmnt{%
  We can also accommodate for a related fact that even counterfactual outcomes
  matter: if you choose an option $[0.9 ? \$1 million : \$0]$, then an outcome
  of $\$0$ will not be regarded as neutral; you will be thoroughly disappointed.
  If there was an alternative sure gain, even it was quite small (say,
  $\$1000$), you will probably regret your choice. Avoidance of regret and
  disappointment are psychologically important, but cannot be explained by
  prospect theory or rational choice theory \cite[287f.]{kahneman11thinking}.%
} %

The problematic type of reference dependence is related to so-called
\textbf{framing effects}. In experiments, people's choices can systematically
depend on how one and the same decision problem is described. When presented
with a hypothetical situation in which 1000 people are in danger of death, and a
certain act would save exactly 600 of them, subjects are more favourable towards
the act if it is described in terms of `600 survivors' than if it is described
in terms of `400 deaths'. In prospect theory, the difference might be explained
by a change in reference point: if the outcome is described in terms of
survivors, it is classified as a gain; if it is described in terms of deaths, it
is classified as a loss.

In principle, our liberal model could explain the relevance of the description.
Perhaps people assign basic value to choosing options \emph{that have been
  described in terms of survivors} rather than in terms of deaths. On
reflection, however, most people would certainly deny that the verbal
description of an outcome is of great concern to them. As in the case of
decision weights, a more adequate model would arguably have to take into account
our incomplete grasp of a verbally described scenario. When hearing about
survivors, we focus on a certain attribute of the outcome, on all the people who
are saved. This attribute is desirable. When hearing about deaths, a different,
and much less desirable, attribute of the same outcome becomes salient.

\cmnt{%
  But is this a failure in decision making, and should we make room
  for it in a descriptively adequate theory of decisions? I'm not
  sure. I suspect it is rather a failure in processing the information
  about the options. Under the positive description (chance of
  success), you really do consider the outcome as more desirable than
  under the negative description. So the problem isn't that you go
  wrong when putting your beliefs and desires into actions. Rather,
  you go wrong when judging the value of a verbally described
  situation. Admittedly, the issue is subtle: your calculation of EUs
  is part of your conforming to decision theory, and isn't this where
  you made the mistake, since you didn't correctly calculate the
  utility of one of the outcomes? But perhaps the more basic reason is
  that you didn't fully understand the content of the outcome you were
  told about. It's not like you were perfectly aware of the relevant
  \emph{proposition}, but sloppy when calculating its utility. The
  presentation-dependence shows that you weren't clear about what
  exactly the outcome would involve: you heard something about success
  and imagined the good case (or the lives saved etc.) and quickly
  gave a value to that, without realizing what else is entailed by the
  presentation. This is less a failure in acting than in
  understanding.
} %

Ideal agents always weigh up all attributes of every possible outcome. Real
agents arguably don't do that, as it requires considerable cognitive effort.%
\cmnt{%
  One of Gigerenzer's heuristics is to consider only the most important
  attribute of any given outcome.%
} %
As a result, the attributes we consider depend on contextual clues such as
details of a verbal description. Some recent models of decision making take this
kind of attribute selection into account.


\section{Imprecise credence and utility}\label{sec:imprecise}

% I'm going to toss three dice. Would you rather get £1000 if the sum of the
% numbers on the dice is at least 10 or if all three numbers are different? You'd
% probably need some time to give a final answer. You know that the six possible
% results for every dice have equal probability, and that the results are
% independent. But it takes some effort to figure out which of the two events I
% described is more likely.

% A real agent's cognitive system can't explicitly store a credence and utility
% for every proposition. It can only store a limited number of
% \textbf{constraints} on credences and utilities. A constraint rules out some
% credences and utilities, but not others. That outcomes of die tosses are
% probabilistically independent is a constraint; it entails that the probability
% of three sixes is the product of the probabilities for the individual dice:
% $\Cr(\text{Six}_1 \land \text{Six}_2 \land \text{Six}_3) = \Cr(\text{Six}_1)\cdot \Cr(\text{Six}_2)\cdot \Cr(\text{Six}_3)$,
% but it does not fix what these probabilities are.

% Will the constraints stored by an agent's cognitive system always be rich enough
% to determine a unique credence and utility function? Perhaps not. There might
% well be questions on which you don't have a settled opinion, even in principle,
% after ideal reflection. Or suppose you don't have time for lengthy reflection,
% or you're too tired. It might be wrong to model your attitudes by a single,
% precise credence function, and a single, precise utility functions.

One respect in which our model is often found unrealistic is that its credences
and utilities are too precise. What is your credence that there will be a
nuclear war before 2100? Is it 0.31832? Or 0.20993? Any such answer may seem
wrong. You haven't made up your mind up to the 5th (let alone the 500th) decimal
point.

Across several disciplines, researchers have developed models that don't assume
unique and precise credences and utilities. On this approach, your credence in a
nuclear war might be given by a whole range of numbers -- perhaps by the
interval [0.2, 0.5] that contains all numbers from 0.2 up to 0.5.

If we want to specify rational norms for such ``imprecise'' credences, it helps
to assume that they are determined by a set of precise credence functions. We
would model your imprecise belief state by a set $\mathbb{C\!r}$ of credence
functions that assign different numbers to the nuclear war hypothesis. For each
number in [0.2, 0.5], there would be a member of $\mathbb{C\!r}$ that assigns this
number to the nuclear war hypothesis. We can then implicitly constrain your
imprecise credences by saying that each member of $\mathbb{C\!r}$ should satisfy
the Kolmogorov axioms.

% The idea is often motivated by consideration like the following. How likely do
% you think it is that it will snow in Toronto on 7 January 2041? If someone
% suggested the probability is 80\%, you might say that's too high; 5\% seems too
% low. But 20\% might seem just as plausible to you as 21\%. It would therefore be
% wrong to model your state of mind by single and precise probability. Rather, we
% should say that your credence is a whole range of numbers, like so:
% \[
%   \Cr(\emph{Snow}) = [0.1, 0.5].
% \]
% Here, `$[0.1, 0.5]$' denotes the range of all numbers from 0.1 to 0.5;
% it is the range of all numbers your refined credence functions assign
% to \emph{Snow}.

% \cmnt{%
%   Suppose you similarly haven't made up your mind about whether
%   Leonardo da Vinci liked porridge: $\Cr(\emph{Porridge}) =
%   [0.2,0.7]$. Nonetheless, you might reasonably treat the two
%   propositions as independent. (That might be one of the constraints
%   encoded in your belief state.) If we model your beliefs in terms of
%   credence intervals, we seem to lose the information about
%   independence.%
%   \cmnt{%
%     Recall that \emph{Snow} and \emph{Porridge} are probabilistically
%     independent just in case $\Cr(\emph{Snow}/\emph{Porridge}) =
%     \Cr(\emph{Snow})$. Plausibly, $\Cr(\emph{Snow} \land
%     \emph{Porridge}) = [0.01,0.25]$, so by the ratio formula, we'd
%     need to assume that $\frac{[0,01,0.25]}{[0.2,0.7]} = [0.1,0.5]$.%
%   } %
% } %

% You should be skeptical about this line of argument. `Probability' in English
% almost always means objective probability. When asked about the probability of
% \emph{Snow}, it is natural to interpret the question as concerning a certain
% objective quantity -- something you could perhaps find out by developing a
% sophisticated weather model. Credence, on the Bayesian conception, is not belief
% about objective probability. It is simply strength of belief. You could not find
% out your credence in \emph{Snow} by developing a sophisticated weather model.

% Nonetheless, there are reasons to extend the Bayesian conception of credence to
% allow for sets of credence functions. For example, suppose we measure (or
% define) an agent's credences and utilities in terms of her preferences. Various
% representation theorems show that if the agent's preferences satisfy certain
% axioms, then the preferences are represented by a \emph{unique} credence and
% utility function (except for the conventional choice of zero and unit for
% utilities). Giving up uniqueness then means that the agent violates one or more
% of the axioms. And it is not implausible that real agents do violate some of
% these axioms.

% Intuitively, each member of $\mathbb{Cr}$ is a \emph{refinement} or
% \emph{precisification} of your indeterminate state of mind.

We can adopt a similar account of utility, replacing our single utility function
$U$ by a set of utility functions $\mathbb{U}$.

On a preference-based approach, ``imprecise'' credences and utilities naturally
arise through violations of the Completeness axiom. Completeness says that for
any propositions $A$ and $B$, you either prefer $A$ to $B$, or you prefer $B$ to
$A$, or you are indifferent between $A$ and $B$. This is trivial if we define
preference in terms of choice. Indeed, in a forced choice between $A$ and $B$,
you will inevitably choose either $A$ or $B$; even indifference can be ruled
out. But we've seen that if we want to measure credence and utility in terms of
preference, then the relevant preference relation can't be directly defined in
terms of choices. Once we take a step back from choice behaviour, it is
conceivable that you neither prefer $A$ to $B$, nor $B$ to $A$, and yet you're
not indifferent between the two. You simply haven't made up your mind. The two
propositions seem roughly ``on a par'', but you wouldn't say they are exactly
equal in value.

% Chang thinks parity is different from incomparability. I don't think so.

For example, would you rather lose your capacity to hear or your capacity to
walk? You may well have no clear preference, even after considerable reflection.
Does this mean that you're exactly indifferent? Not necessarily. If you were,
you should definitely prefer losing the capacity to hear \emph{and getting £10}
to losing the capacity to walk. In reality, the added £10 may not make a
difference. 

\begin{exercise}{2}
  Suppose we define `$A\sim B$' as `not $A \succ B$ and not $B \succ A$'. It is
  then logically guaranteed that either $A \succ B$ or $B\succ A$ or $A \sim B$.
  But Transitivity might fail, if you haven't fully made up your mind. Explain
  why.
\end{exercise}

Even if we give up completeness, however, we might still require
\textbf{completability}. We might want to say that if an agent's preferences
violate, say, Ramsey's axioms because they fail to rank certain options, then
there is a refinement of their preferences, filling in the missing rankings,
that does satisfy the axioms. Ramsey's representation theorem then implies that
the agent's preferences are represented by a \emph{set} of credence and utility
functions.

Allowing for a set of credence and utility functions requires some changes to our
model. How should a set of credence functions be revised when new information
comes in? How should an agent choose based on a set of credence and utility
functions? Both questions raise serious problems.

The most obvious answer to the first question is that if an agent has a set of
credence functions $\mathbb{C\!r}$ and receives total evidence $E$, then her new
set of credence functions should result from $\mathbb{C\!r}$ by conditionalising
each member of $\mathbb{C\!r}$ on $E$.

One problem with this answer is that this process is, in general,
computationally \emph{harder} than conditionalising a single probability
measure. In this respect, our model has become less realistic, not more.

Here is another problem. Suppose I have an urn containing 2 balls, one of which
is white. The other is either white or red. You have no opinion about how the
other ball's colour: your belief state $\mathbb{C\!r}$ contains all possible
probability assignments to the hypothesis that the other ball is white. Now I
shuffle the urn, draw a ball, and show it to you. The ball is white. If you
conditionalise each member of $\mathbb{C\!r}$ on this information, your belief
state remains unchanged! Your new imprecise credence is still $\mathbb{C\!r}$. It
remains at $\mathbb{C\!r}$ no matter how often I draw a white ball, each time
replacing the previously drawn ball. This seems wrong.

\begin{exercise}{3}
  Explain why seeing a white ball doesn't change $\mathbb{C\!r}$.
\end{exercise}

Let's briefly turn to the other question. How should you choose between some
options if you have a set of credence and utility functions? Suppose option $A$
maximizes expected utility relative to one of your credence and utility
functions, while option $B$ maximizes expected utility according to another.
Should you choose $A$ or $B$? A popular ``permissivist'' answer is that either
choice is acceptable.

\begin{exercise}{2}
  Explain how the preference of $A$ over $B$ and $D$ over $C$ in Ellsberg's
  paradox might be justified by the permissivist approach, without redescribing
  the outcomes.
  % (What is the expected utility of the four options?)
\end{exercise}

But now imagine you are offered two bets $A$ and $B$, one after the other, on a
proposition $H$ to which you don't assign a precise credence. Let's say your
credence in $H$ spans the range from 0.2 to 0.8. Bet $A$ would give you
£1.40 if $H$ and £-1 if $\neg H$. Bet $B$ would give you £-1 if $H$ and £1.40 if
$\neg H$. Assume for simplicity that your utility is precise and proportional to
the monetary payoff. The permissivst account then classifies both bets as
optional: you may take them or leave them. But accepting both bets yields a
guaranteed gain of £0.40. By refusing both bets, you would miss out on a sure
gain.

% Both bets then have an imprecise expected utility of
% between -0.52 and 0.92. (For example, the expected utility of the first bet
% ranges from $0.2 \cdot -1 + 0.8 \cdot 1.40 = 0.92$ to
% $0.8 \cdot -1 + 0.2 \cdot 1.40 = -0.52$.)

\begin{sources}

  A standard textbook on artificial intelligence is Stuart Russell and Peter
  Norvig, \emph{Artificial Intelligence: A Modern Approach} (4th ed., 2020).
  Part IV covers most of the material I have summarized in section \ref{sec:ai}.

  % In the jargon of computer science, maximizing the expected long-run utility is
  % called \textbf{planning}.

  % More on bayes nets?
  
  For evidence that our brains might use some of the tricks AI researchers have
  found see, for example, Samuel Gershman and Nathaniel Daw, ``Perception,
  Action and Utility'' (2012). For a more high-level view on the idea that
  cognitive systems try to approximate the Bayesian ideal, see Thomas Griffiths
  et al, ``Rational Use of Cognitive Resources'' (2015), or Samuel Gershman et
  al, ``Computational rationality: A converging paradigm'' (2015).

  % There is evidence that our brains do process sensory signals roughly in line
  % with \eqref{Bayesint}, and evaluate actions and outcomes roughly in line
  % with \eqref{EUint} (see \cite[296--299]{gershman12perception} for
  % literature). Of course, the two components are not entirely independent (a
  % point emphasised in \cite{gershman12perception}): organism have greater
  % interest in learning about aspects of the world with possibly extreme
  % (positive or negative) utility; thus our sensory processing is often attuned
  % to what matters to us in terms of utility.

  % Sophisticated models have been developed that apply a kind of meta-decision
  % procedure to the choice of a computational approximation of \eqref{Bayesint}
  % and \eqref{EUInt}, as reviewed in \cite[306ff.]{gershman12perception}.
  % Again, one can let the system perform meta-analyses about the amount of
  % effort to put in, e.g. when to stop getting a more precise update.

  % An exciting aspect of these models is that they explain ``fallacies'' not as
  % accidental quirks but as consequences for optimizing with limited resources.
  % See e.g. Griffiths et al 2015. We can also explain why e.g. cognitive load or
  % stress etc. affect rational choice by making it reasonable to delegate more
  % choices to ``model-free'' habitual systems.

  % See also Halpern et al. 2014 for possible explanations of biases.
  
  For a brief overview of prospect theory and related models, motivated by the
  idea of bounded rationality, see Daniel Kahneman, ``A Perspective on Judgment
  and Choice'' (2003). The empirical claims about probabilities, frequencies,
  and reference points in section \ref{sec:prospect-theory} are from Kahneman's
  \emph{Thinking Fast and Slow} (2011).

  For a model of attribute selection in the evaluation of options, see Franz
  Dietrich and Christian List, ``Reason-Based Choice and Context-Dependence''
  (2016).
  % Busemeyer et al. (2006) describe a choice mechanism they call ‘decision field
  % theory’, whereby an agent only ever evaluates one aspect of an option at any
  % one moment in time. Over time, attention shifts stochastically to other
  % aspects of the option, and the evaluation is integrated into the previous
  % evaluation. Once some threshold is reached, a decision is announced.
  There are also models for how to selectively use different aspects of a
  credence function. See Peter Fritz and Harvey Lederman, ``Standard State Space
  Models of Unawareness'' (2015).

  % There is fairly solid evidence that when mentalistic preferences over
  % anything but very coarse-grained objects are elicited, these are typically
  % constructed “on the fly”. See Lichtenstein and Slovic (2006) for an overview
  % and Bettman (1979) for an early proponent. -- This is on how storing basic
  % desires and additivity can simplify.
  
  The
  \href{https://plato.stanford.edu/entries/imprecise-probabilities/}{Stanford Encyclopedia Entry ``Imprecise Probabilities''}
  by Saemus Bradley (2019) provides a good overview of research on the topic of
  section \ref{sec:imprecise}. The urn problem is an instance of ``belief
  inertia''.

  The Ellsberg Paradox was presented in Daniel Ellsberg, ``Risk, Ambiguity, and
  the Savage Axioms'' (1961).

\end{sources}




%%% Local Variables: 
%%% mode: latex
%%% TeX-master: "bdrc.tex"
%%% End:
