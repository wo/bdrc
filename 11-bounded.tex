\chapter{Bounded Rationality}

\section{Models and reality}

We have studied an abstract model of rational agents. The model
assumes that the agents to be modelled have some idea of what the
world might be like, which we represent by a credence function $\Cr$
over a suitable space of propositions. We also assume that the agents
have some goals or values or desires, which we represent by a utility
function $U$ on the same space of propositions. The credence function
is assumed to satisfy the formal rules of the probability calculus; it
is assumed to evolve over time by conditionalizing on sensory
information, and it is assumed to satisfy further constraints such as
the Probability Coordination Principle. The utility function $U$ is
assumed to satisfy Jeffrey's axiom, so that it is jointly determined
by the agent's credences and the basic values assigned to individual
possible worlds. The basic values may in turn be determined by
aggregating subvalues. When the agent faces a choice, she is assumed
to choose an act which maximizes the credence-weighted average of the
utility of the act's possible outcomes.

There are different ways of filling in the details, and so our model
is really a family of models. Should expected utility be understood
causally or evidentially? Should credences satisfy a restricted
version of the Principle of Indifference? Should utilities conform to
localism?  Should they be stationary? And so on.

Each model in this family can be understood either
\textbf{normatively} or \textbf{descriptively}. Understood
normatively, the model would purport to describe an ideal which real
agents should perhaps aspire to. Understood descriptively, the model
would purport to describe the attitudes and choices of ordinary
humans.

It is a commonplace in current economics and psychology that our
models are descriptively inadequate: that real people are not expected
utility maximizers. In itself, this is not necessarily a problem --
not even for the descriptive interpretation of our model. Remember
that ``all models are false''. With the possible exception of the
``standard model'' of particle physics, the purpose of a model is
generally to identify interesting and robust patterns in the
phenomena, not to get every detail right.  Nonetheless, it is worth
looking at how real agents fail to conform to our model, and what we
could change to make it more realistic.

Many examples of people supposedly failing to maximize expected
utility are not really counterexamples to the descriptive adequacy of
our model, since the examples rely on implausible restrictions on the
agent's utility function. As we saw in chapter \ref{ch:risk}, our
model can easily accommodate agents who care about risk or fairness or
feelings of frustration, which explains the common preferences in
Allais's paradox. We can similarly accommodate altruistic behaviour
(section \ref{sec:decision-matrices}), the endowment effect (section
\ref{sec:sources-utility}), or apparent failures of time consistency
(section \ref{sec:separability-time}).

\cmnt{%
  Another fact is that preference orders themselves can vary. They
  clearly vary across long stretches of time: my favourite food is no
  longer what it used to be when I was seven. (We could say that my
  preference is still the same, for I still prefer eating pasta and
  being seven to eating brussels sprouts and being seven, but it is
  not clear that anything is gained from that.) They can also vary in
  shorter intervals -- from morning to evening, from talking to a
  pretty woman to no longer talking to her. Here again we need
  substantive constraints to divide these into ``rational'' changes
  and ``irrational'' changes, and the borderline is fuzzy.
} %

But other phenomena are harder to accommodate. For example, suppose I
offer you £100 for telling me the prime factors of 82,717. You have 10
seconds. Can you do it? Probably not. All you'd have to do to get the
money is utter `181 and 457', which is surely an available
act. Moreover, that `181 and 457' is the correct answer logically
follows from simpler facts of which you are highly confident. By the
rules of probability, you should therefore be confident that `181 and
457' is the correct answer. To maximize expected utility (assuming
you'd like to get the £100), you should therefore utter these
words. Yet you don't.

\begin{exercise}
  Explain why, if some proposition $C$ logically follows from two
  other propositions $A$ and $B$, and $\Cr(A) > 0.9$ and $\Cr(B) >
  0.9$, then $\Cr(C) > 0.81$. $\star\star$
\end{exercise}

In 1913, Ernst Zermelo proved that in the game of chess, there is
either a strategy for the starting player, White, that guarantees
victory no matter what Black does, or there is such a strategy for
Black, or there is a strategy for either player to force a
draw. Consequently, if two ideal Bayesian agents sat down to a game of
chess, and their only interest was in winning, they would agree to a
draw or one of them would resign before the first move. Real people
don't play like that.

Another obvious respect in which real people deviate from our model is
that real people often forget. In section \ref{sec:why-meu}, I
mentioned forgetting to buy soap in the shop. That's a temporary kind
of forgetting, because you later remember what you forgot. We also
often lose information completely. As an ideal Bayesian agent, you
would still know what you had for breakfast on 12 June 2006.

\begin{exercise}
  Show that if $\Cr_t(A) = 1$, and the agent conditionalizes on
  information $E$ with $\Cr_t(E) > 0$, then $\Cr_{t+1}(A) =
  1$. (Conditionalization was introduced in
  section \ref{sec:conditionalization}.) $\star\star$
\end{exercise}

\cmnt{%
  Other issues are less clear. E.g., it is widely claimed that real
  agents are not ``logically omniscient'', meaning that they are not
  certain of every logical truth, in violation of Kolmogorov's axiom
  2. But it is not obvious what it would mean for an agent to have
  credence less than 1 in a logical truth, and therefore it is also
  not obvious whether people actually have such attitudes. Indeed, if
  `credence' is a technical term defined by our model, then so far it
  is trivially true that everyone is logically omniscient. We would
  need an alternative model.
} %

So it's plausibly true that our model doesn't fit real agents in
certain respects. There is indirect evidence for this from research on
artificial intelligence. The type of model we have studied is well
known in these quarters, and forms the background for much recent
research.%
\cmnt{%
  see RN.%
} %
Yet the model turns out to be computationally intractable. Real agents
with limited cognitive resources, it seems, couldn't possibly conform
to our model.


\section{Avoiding computational costs}

Before we look at ways of making our model more realistic, I want to
address another common misunderstanding.

Suppose you walk back to shop to buy soap. At any point on your way,
you could decide to turn around, or start running, or check if your
shoe laces are tied, or mentally compute $181 + 457$, or start humming
the national anthem, or utter `Age quod agis', and so on. There are
literally millions of things you could do. Many of these would lead to
significantly different outcomes, especially if you consider long-term
consequences. (Hitler almost certainly would not have existed if hours
or even months before his conception, his mother had decided to run
rather than walk to buy soap.) Some people take the MEU Principle to
imply that at each point of your walk to the shop, you should
explicitly consider all your options, envisage all their possible
outcomes, assess their utility and probability, and on that basis
compute their expected utility. This is clearly unrealistic and
infeasible.

But the MEU Principle requires no such thing. The MEU Principle says
that rational agents choose acts that maximize expected utility; it
specifies \emph{which acts} an agent should choose, given her beliefs
and desires. It says nothing about the internal process that leads
agents to these choices. It definitely doesn't say that the agent must
explicitly consider all her options and compute expected utilities.

\begin{exercise}
  The opposite is closer to the truth. Suppose an agent has a choice
  between turning left ($L$), turning right ($R$), and sitting down to
  compute the expected utility of $L$ and $R$ and then choosing
  whichever comes out best. Let $C$ be that third option. If computing
  expected utilities involves some costs in terms of effort or time,
  then either $L$ or $R$ generally has greater expected utility than
  $C$. Explain why. $\star\star$
\end{exercise}

\cmnt{%
  The reason is simple. Let $A_1,\ldots,A_n$ be your ``basic''
  options, those that don't involve computing expected utilities. In
  addition to these, we assume, you have the option $C$ of computing
  the expected utilities of $A_1,\ldots,A_n$ and then choosing
  whichever of them comes out best. Let $A_i$ be the best of the basic
  options. If you choose $C$, you are guaranteed to get the same
  outcome as if you choose $A_i$ directly, except that you've also
  spent some time and effort on computing expected utilities. Usually,
  that will ensure that whatever outcome you get from $C$ is worse
  than what you would have gotten from $A_i$. Thus $A_i$ has greater
  expected utility than $C$, and you should not choose $C$.
  
  The argument just outlined is not fully general. There are unusual
  cases in which $C$ would have greater expected utility than directly
  choosing $A_i$, even though the computation has negative
  (sub)value. What do such cases look like? $\star\star$

  So unless they get some direct value from computing expected
  utilities, ideal Bayesian agents don't compute expected
  utilities. Instead they directly choose whichever option maximizes
  expected utility.
  
  That may seem strange. How is the agent supposed to arrive at her
  choice of the best option, if not by figuring out that it maximizes
  expected utility? By accident? That doesn't look rational. By magic?
  
  Part of the answer is that ideal Bayesian agents have no need to
  compute anything: if they are certain of some facts, and these facts
  entail other facts, then the probability axioms entail that they are
  also certain of the other facts. For ideal Bayesian agents,
  computations are a costly process of arriving at results that are
  already known.
} %

\cmnt{%
  So ideal Bayesian agents generally don't compute expected
  utilities. That may seem puzzling. How can you reliably choose an
  act with greatest expected utility if you don't compute expected
  utilities?
} %

So the MEU Principle does not require calculating expected
utilities. But that raises a puzzle. An agent who conforms to our
model always chooses acts with greatest expected utility. How is she
supposed to do that without calculating? It doesn't seem rational to
choose one's acts randomly and maximize expected utility out of sheer
luck.

The answer (or at any rate a part of the answer) to this puzzle is
that there are many circumstances in which simple decision methods
reliably lead to choices that maximize expected utility. As the
psychologist Gerd Gigerenzer once pointed out, if you want to catch a
flying ball, an efficient alternative to computing the ball's
trajectory -- which is generally intractable -- is to move around in
such a way that the angle between you and the ball remains within a
certain range. This ensures that you'll eventually stand where the
ball will arrive. The point is that one can often fall back on simple
heuristics in order to reliably perform a certain task. If you desire
to catch the ball, following Gigerenzer's heuristic will maximize
expected utility. You don't need to consciously compute anything, and
you don't need to conceptualize what you're doing as maximizing
expected utility.

\begin{exercise}
  Suppose you're a musician in the middle of a performance. Trying to
  compute the expected utility of all the notes you could play next
  would probably derail your play. Even if it wouldn't, it would
  completely change your experience of playing, probably for the
  worse. Give another example where conceptualizing one's acts as
  maximizing expected utility would undermine the value of performing
  the acts. $\star$
\end{exercise}

One reason why many decision problems don't require sophisticated
computations is that a certain act clearly dominates all the
others. Whether that is the case depends on the agent's utility
function. So you can reduce the computational costs of decision-making
by tweaking your utilities. For example, suppose you assign
significant (sub)value to obeying orders. Doing whatever you're
ordered to do is then a reliable way of maximizing expected utility,
and it requires very little cognitive effort. Similarly if you value
imitating whatever your peers are doing.

\cmnt{%
  Endowing individuals with a desire for social conformity might be an
  evolutionary trick to ensure that people make reasonably simple
  choices without having to reason too much. (Conformity has other
  evolutionary advantages as well.)
} %

Our capacity for planning and commitment can also be seen in this
light. Before you went to the shop, you probably decided to go to the
shop. The direct result of that decision was an intention to go to the
shop.%
%
\cmnt{%
  Somewhat confusingly, according to our model, this was not a
  decision to go to the shop. That's because, at the time, going to
  the shop was not strictly speaking an available act, since was
  partly outside your control. Someone could have prevented you from
  leaving the house. You could have gotten into an accident while
  crossing the street. Your future self could have been seduced by the
  sight of the pub and stopped for a few beers. When you decided to go
  to the shop, the ``act'' you chose was an act of commitment or
  planning. You formed the intention of going to the shop.%
} %
%
Once an intention or plan if formed, we assign value to executing the
plan. Revising a plan or overturning a commitment has negative
(sub)value. Consequently, once you've formed an intention, simply
following it reliably maximizes expected utility. You don't need to
think any more about what to do unless you receive surprising new
information or your basic values suddenly change. (This is true even if
you made a mistake when you originally formed the intention.)

Habits can play a similar role. Most of us spend little effort
deciding whether or not to brush our teeth in the morning; we do it
out of habit. Habitual behaviour is computationally cheap, and it can
reliably maximize expected utility -- especially if we assign
(sub)value to the habitual behaviour. And we do, at least on a
motivational conception of desire: habits motivate.

The upshot is that various cognitive strategies that are often
described as alternatives to computing expected utilities -- habits,
instincts, heuristics, etc.\ -- may well be efficient
techniques for maximizing expected utility. Far from ruling out such
strategies, our model actually predicts that we should use them.

\cmnt{%
  Still, one might complain that our model lacks detail.  We could
  model pointless habitual behaviour as simply MEU, but to understand
  why someone might have the relevant utilities, it is useful to
  develop a more complex model that distinguishes between deliberate
  and habitual behaviour. (Such ``two systems'' models are common both
  in psychology and neuroscience; see e.g.\ Dickinson A. Actions and
  habits: the development of behavioural autonomy. Philosophical
  Transactions of the Royal Society B: Biological Sciences. 1985;
  308:67–78 and 2 system theory.)
} %

An example in which something like this might play a role is
Ellsberg's Paradox, another classical ``counterexample'' to the MEU
Principle.

\begin{example}[Ellsberg's Paradox]
  An urn contains 300 balls. 100 of the balls are red, the others are
  green or blue, in unknown proportion. A ball is drawn at random from
  the urn. Which of the following two gambles do you prefer?
  % 
  \begin{center}
  \begin{tabular}{|r|c|c|c|}\hline
    \gr & \gr Red & \gr Green & \gr Blue \\\hline
    \gr $A$ & £1000 & £0 & £0 \\\hline
    \gr $B$ & £0 & £1000 & £0  \\\hline
  \end{tabular}
  \end{center}
  %
  Next, which of $C$ and $D$ do you prefer?
  % 
  \begin{center}
  \begin{tabular}{|r|c|c|c|}\hline
    \gr & \gr Red & \gr Green & \gr Blue \\\hline
    \gr $C$ & £1000 & £1000 & £0 \\\hline
    \gr $D$ & £0 & £1000 & £1000 \\\hline
  \end{tabular}
  \end{center}
\end{example}

Many subjects prefer $A$ to $B$ and $D$ to $C$. Like in Allais's
Paradox, there is no way of assigning utilities to the monetary
outcomes that supports these preferences.

\begin{exercise}
  Assume the outcomes in Ellsberg's paradox are described correctly
  and you prefer more money to less. By the Probability Coordination
  Principle, $\Cr(\emph{Red}) = \nicefrac{1}{3}$. What would your
  credences in \emph{Green} and \emph{Blue} have to look like so that
  $EU(A) > EU(B)$? What would they have to look like so that $EU(D) >
  EU(C)$? $\star$
\end{exercise}

In Ellsberg's Paradox, risk aversion doesn't seem to be at issue. What
makes the difference is that you know the objective probability of
winning for options $A$ and $D$: it is \nicefrac{1}{3} for $A$ and
\nicefrac{2}{3} for $D$. But you don't know the objective probability
of winning with $B$ and $C$, since you have too little information
about the non-red balls. 

Why does that matter? One explanation is that people simply prefer
lotteries (in which the outcomes have known objective
probabilities) to uncertain prospects (in which only subjective
probability can be given to the outcomes). With such a utility
function, the outcome wrongly labelled `£1000' in $A$ is actually
better than the corresponding outcome in $C$, because only the former
involves having chosen a lottery.

\begin{exercise}
  The explanation of the Ellsberg preferences that I just outlined
  makes the preferences conform to the MEU Principle by redescribing
  the outcomes. Is the redescription global or local in the sense of
  chapter \ref{ch:risk}? $\star$
\end{exercise}

But why would agents prefer lotteries? Perhaps because such a
preference can reduce computational costs. If you know the objective
probabilities of the outcomes, it is easy to figure out the credence
you should give to the outcomes: it should match the objective
probabilities. If you don't know the objective probabilities, a lot
more work may be required to figure out the extent to which your total
evidence supports the occurrence of the various outcomes. In
Ellsberg's Paradox, $\Cr(\text{Red})$ is a easier to figure out than
$\Cr(\text{Green})$ and $\Cr(\text{Blue})$. If you have a preference
for lotteries, you don't need to figure out $\Cr(\text{Green})$ and
$\Cr(\text{Blue})$: from eyeballing the options, you can already see
that the expected monetary payoff of $A$ and $B$ is approximately the
same (ditto for $C$ and $D$); a preference for lotteries then clearly
favours $A$ (and $D$).


\section{Reducing computational costs}

I will now review a few ideas from theoretical computer science for
rendering our models computationally tractable.

Imagine we are designing an artificial agent, with a probabilistic
representation of her environment and a number of goals or
desires. Let's assume we want the agent to have credences and
utilities in 50 logically independent propositions
$A_1,\ldots,A_{50}$ (an absurdly small number).
How large of a database do we need for that?

You might think that we need 50 records for the probabilities and 50
for the utilities. But we generally can't compute $Cr(A \land B)$ or
$\Cr(A \lor B)$ from $\Cr(A)$ and $\Cr(B)$. Nor can we compute $U(A
\land B)$ or $U(A \lor B)$ from $U(A)$ and $U(B)$. In order to fix the
agent's entire credence and utility functions (without further
assumptions about these functions), we need to store at least the
probability and utility they assign to every ``possible world'' --
that is, to every maximally consistent conjunction of
$A_1,\ldots,A_{50}$ and their negations.

\begin{exercise}
  Explain why the probability of every proposition that can be defined
  in terms of $A_1,\ldots,A_{50}$ can be computed from the probability
  assigned to the possible worlds. Then explain why the utility of all
  such propositions can be computed from the probability and utility
  assigned to the worlds. $\star\star\star$
\end{exercise}

\cmnt{%
  Since every proposition that can be defined in terms of
  $A_1,\ldots,A_{50}$ is equivalent to a disjunction of these possible
  worlds, Kolmogorov's axiom (iii) and Jeffrey's Axiom ensure that we
  can compute all probabilities and utilities from the assignment to
  possible worlds.
} %

There are $2^{50} = 1,125,899,906,842,624$ maximally consistent
conjunctions of $A_1,\ldots,A_{50}$ and their negations.  So we need a
database with $2,251,799,813,685,248$ records. (I'm exaggerating a
little. Once we've fixed the probability of the first
1,125,899,906,842,623 worlds, the probability of the last world is 1
minus the sum of the others, so we really only need
$2,251,799,813,685,247$ records.)

We'll need to buy a lot of hard drives for our agent if we want to
store 2 quadrillion floating point numbers. Worse, updating all these
records in response to sensory information, or computing expected
utilities on their basis, will take a very long time, and use a large
amount of energy.

In chapter \ref{ch:separability}, we've encountered two tricks for
simplifying the representation of an agent's basic value function --
the utility function for individual worlds. First, if the agent cares
about some features of the world and not about others, it is enough to
store the agent's value for sets of worlds that agree in all respects
that matter to the agent; all worlds within such a set will have the
same value as the set itself (section
\ref{sec:construction-value}). For example, if our agent only cares
about the possible combinations of 20 among the 50 propositions
$A_1,\ldots,A_n$, we only need to store $2^{20}$ values. Second, and
more dramatically, if the agent's preferences are separable, we can
cut down the number of utility records from $2^{20}$ to $2 \times 20 =
40$, because the value of any combination of the 20 propositions and
their negations can be determined by adding up the relevant subvalues
(section \ref{sec:additivity}).

Similar tricks are available for the agent's credence
function. Mirroring the first trick, we could explicitly store only
the agent's credence in certain sets of possible worlds, and assume
that her credence is distributed uniformly within these sets. The
trick can be extended to non-uniform distributions. For example,
suppose our agent has imperfect information about how far she is from
the next charging station. Instead of explicitly storing a probability
for every possible distance (1 m, 2 m, 3 m, \ldots), we might assume
that the agent's credence over these possibilities follows a Gaussian
distribution, which can be specified by two numbers (mean and
variance). Researchers in artificial intelligence make heavy use of
this trick.

An analogue of separability, for credences, is probabilistic
independence. If $A$ and $B$ are probabilistically independent, then
$\Cr(A\land B) = \Cr(A) \times \Cr(B)$. If all the 50 propositions
$A_1,\ldots,A_{50}$ are mutually independent, then we can fix the
probability of all possible worlds and therefore of all logical
combinations of the 50 propositions by specifying their individual
probability.

Independence is often plausible. Whether the next charging station is
100 meters away plausibly doesn't depend on whether the outside
temperature is above 20\celsius. But for many other propositions,
independence is implausible. On the supposition that the temperature
is above 20\celsius, it may well be more likely that the window is
open, or that there are people on the street; so
$\Cr(\emph{Open}/\emph{Warm}) > \Cr(\emph{Open})$, and
$\Cr(\emph{People}/\emph{Warm}) > \Cr(\emph{People})$. So we can't
assume probabilistic independence across all the 50 propositions
$A_1,\ldots,A_{50}$.

Even where independence fails, however, we often have
\textbf{conditional independence}. For example, if warm temperatures
make it more likely that the window is open and that there are people
on the street, then an open window is also evidence that there are
people on the street: $\Cr(\emph{People}/\emph{Open}) >
\Cr(\emph{People})$.  So \emph{People} and \emph{Open} are not
independent. However, \emph{on the supposition that it is warm
  outside}, the window being open may no longer increase the
probability of people on the street:
\[
\Cr(\emph{People}/\emph{Open} \land \emph{Warm}) = \Cr(\emph{People}/
\emph{Warm}).
\]
In that case, we say that \emph{People} and \emph{Open} are
independent \emph{conditional on} \emph{Warm}.

Now consider the possible combinations of \emph{Warm}, \emph{People},
\emph{Open} and their negations. By the probability calculus (compare
exercise \ref{e:chain-rule}),
\[
\Cr(\emph{Warm} \land \emph{People} \land \emph{Open}) = 
\Cr(\emph{Warm}) \times \Cr(\emph{Open}/\emph{Warm}) \times \Cr(\emph{People}/\emph{Open} \land \emph{Warm}).
\]
By the above assumption of conditional independence, this simplifies to
\[
\Cr(\emph{Warm} \land \emph{People} \land \emph{Open}) = 
\Cr(\emph{Warm}) \times \Cr(\emph{Open}/\emph{Warm}) \times \Cr(\emph{People}/\emph{Warm}).
\]
In general, with the assumption of conditional independence, we can
fix the probability of all combinations of \emph{Warm}, \emph{People},
\emph{Open}, and their negations by specifying the probability of
\emph{Warm}, the probability of \emph{People} conditional on
\emph{Warm} and on $\neg\emph{Warm}$, and the probability of \emph{Open}
conditional on \emph{Warm} and on $\neg\emph{Warm}$.  This reduces the
number of required records from $2^3-1 = 7$ to 5, which may not look
all that impressive, but the method really pays off if more than three
propositions are involved.

The present technique of exploiting conditional independence to
simplify probabilistic models is known under the heading of
\textbf{Bayesian networks} (or \textbf{Bayes nets}, for short). Bayes
nets have proved useful in wide range of applications.

A special case of Bayes nets%
\cmnt{%
  called \textbf{Dynamic Decision Networks}%
} %
are commonly used in artificial intelligence to model decision-making
agents. A decision-making agent needs not only information about the
present state of the world, but also about the future. We can model a
whole history of states as a sequence $\t{S_1,S_2,S_3,\ldots}$, where
$S_1$ is a particular hypothesis about the present state, $S_2$ about
the next state, and so on. If there are 100 possible states at any
given time, there will be $100^{10} = 100,000,000,000,000,000,000$
possible histories with length 10. Instead of storing individual
probabilities for all these histories, it helps to assume that a later
state (probabilistically) depends only on its immediate predecessor,
so that $\Cr(S_3/S_1 \land S_2) = \Cr(S_3/S_2)$. This is known as the
\textbf{Markov assumption}. It reduces the number of records we'd need
to store from $100^{10}$ to $990,100$.%
\cmnt{%
  $100 + (100\times 100)\times 99 \approx 1000000$ %
} %
\cmnt{%
  It also simplifies belief updates through
  conditionalization. Assuming that sensory information only carries
  direct information about the present state of the world, we only
  need to store $\Cr(E/S)$ for any evidence proposition $E$ and state
  $S$.%
} %

To further simplify the task of decision-making, computer scientists
typically assume that basic values are stationary and separable across
times, so that the value of a history of states is a discounted sum of
the subvalue for individual states. To specify the whole utility
function, we then only need to store the discounting factor $\delta$
and 100 values for the individual states. The task of
conditionalization can also be simplified, by assuming that sensory
evidence only contains direct information about the present state of
the world, rather than entire histories.

These simplifications define what computer scientists call a
`\textbf{POMDP}': a \textbf{Partially Observable Markov Decision
  Process}. There is a simple recursive algorithm for computing
expected utilities in POMDPs.%
\cmnt{%
  In the jargon of computer science, maximizing the expected long-run
  utility is called \textbf{planning}.  } %

\cmnt{%
  There is evidence that our brains do process sensory signals roughly
  in line with \eqref{Bayesint}, and evaluate actions and outcomes
  roughly in line with \eqref{EUint} (see
  \cite[296--299]{gershman12perception} for literature). Of course,
  the two components are not entirely independent (a point emphasised
  in \cite{gershman12perception}): organism have greater interest in
  learning about aspects of the world with possibly extreme (positive
  or negative) utility; thus our sensory processing is often attuned
  to what matters to us in terms of utility.
} %

\cmnt{%
  \begin{exercise}
    Explain one respect in which real decision situations are not
    adequately modelled by POMDPs.
  \end{exercise}
} %

In practice, even these simplifications generally don't suffice to
make conditionalization and expected utility maximization
tractable. Further simplifications are needed. For example, it often
helps to ``myopically'' ignore states in the distant future and let
the agent maximize the expected utility for the next few states
only. In addition, various techniques have been developed that allow
an efficient \emph{approximate} computation of expected utilities and
posterior probabilities. Such techniques are often supplemented by a
meta-decision process which lets the system choose a level of
precision: when a lot is at stake, it is worth spending more effort on
getting the computations right.

\cmnt{%
  Sophisticated models have been developed that apply a kind of
  meta-decision procedure to the choice of a computational
  approximation of \eqref{Bayesint} and \eqref{EUInt}, as reviewed in
  \cite[306ff.]{gershman12perception}. Again, one can let the system
  perform meta-analyses about the amount of effort to put in,
  e.g. when to stop getting a more precise update.

  ``Metalevel analyses have been aimed at endowing computational
  systems with the ability to make expected utility decisions about
  the ideal balance between effort or delay and the quality of actions
  taken in the world. The use of such rational metareasoning plays a
  central role in decision-theoretic models of bounded rationality
  (10–14)''.

  Note that if computational cost is a form of utility then once again
  we actually don't have a violation of MEU, but that's not a very
  illuminating account. (The general picture that emerges is that our
  model is largely correct, but rather uninformative for bounded
  agents.)
} %

While originating in theoretical computer science, these models and
techniques have in recent years had a great influence on our models of
human cognition. There is evidence that when our brain processes
sensory information or decides on a motor action, it employs the same
techniques computer scientists have found useful in approximating the
Bayesian ideal. Thus it has been suggested that various quirks of
human perception and decision-making are consequences of the shortcuts
our brain uses to approximate conditionalization and computing
expected utilities.%
\cmnt{%
  The family of techniques known as \textbf{predictive processing} has
  even been taken to offer an entire new perspective on many aspects
  of human cognition.%
} %

\cmnt{%
  In computer science they are usually approximated either by Monte
  Carlo simulation or by choosing a more tractable distribution $Q$
  that is similar to $P$ (as measured by KL
  distance). \cite[302f.]{gershman12perception} reviews experimental
  findings that suggest that the brain also uses some such
  approximation. For example, abrupt changes of response over the
  course of continuous learning are well explained by an underlying
  Monte Carlo approximation for \eqref{Bayesint} and \eqref{EUint}.
} %

\cmnt{%
  An exciting aspect of these models is that they explain
  ``fallacies'' not as accidental quirks but as consequences for
  optimizing with limited resources. See e.g. Griffiths et al 2015. We
  can also explain why e.g. cognitive load or stress etc. affect
  rational choice by making it reasonable to delegate more choices to
  ``model-free'' habitual systems.

  Halpern et al. 
} %

\section{``Non-expected utility theories''}

Meanwhile, researchers at the intersection of psychology and economics
have also tried to develop more realistic models of
decision-making. The most influential of these alternatives is
\textbf{prospect theory}, developed by Daniel Kahneman and Amon
Tversky.

Prospect theory has to be understood on the background of a highly
restricted version of decision theory still popular in economics. The
highly restricted theory only deals with choices between lotteries,
with known objective probabilities. The outcomes of these lotteries
are identified with monetary wealth or commodity bundles. (Even local
feelings of frustration or regret are excluded.)  Finally, it is
assumed that agents always prefer more money or goods, with declining
marginal utility. When you find social scientists discuss ``Expected
Utility Theory'', this highly restricted theory is what they usually
have in mind. Prospect theory now proposes four main changes.

1. \emph{Reference dependence}. According to prospect theory, agents
classify possible outcomes into gains and losses, by comparing the
outcomes with a contextually determined reference point. Outcomes
better than the reference point are modelled as having positive
utility, outcomes worse than the reference point have negative
utility.

2. \emph{Diminishing sensitivity}. Prospect theory holds that both
gains and losses have diminishing marginal utility: the same objective
difference in wealth makes a larger difference in utility near the
reference point than further away, on either side. For example, the
utility difference between a loss of £100 and a loss of £200 is
greater than that between a loss of £1000 and a loss of £1100. This
predicts that people are risk averse about gains but risk seeking
about losses: they prefer a sure gain of £500 to a 50 percent chance
of £1000, but they prefer a 50 percent chance of losing £1000 to
losing £500 for sure.

3. \emph{Loss aversion}: According to prospect theory, people are
more sensitive to losses than to gains of the same magnitude. For
example, the utility difference between a loss of £100 and a loss of
£200 is greater than that between a gain of £200 and a gain of
£100. This explains why many people turn down a lottery in which they
can either win £110 or lose £100, with equal probability.

4. \emph{Probability weighting}. According to prospect theory, the
outcomes are weighted not by their objective probability, but by
transformed probabilities known as `decision weights' that are meant
to reflect how seriously people take the relevant states in their
choices. Decision weights generally overweight low-probability
outcomes. Thus probability 0 events have weight 0, probability 1
events have weight 1, but in between the weight curve is steep at the edges
and flatter in the middle: probability 0.01 events might have weight
0.05, probability 0.02 events weight 0.08, \ldots, probability 0.99
events weight 0.92. Among other things, this is meant to explain why
people play the lottery, and why they tend to pay a high price for
certainty: they prefer a settlement of £90000 over a trial in which
they have a 99\% chance of getting £100000 but a 1\% chance of
getting nothing.%
\cmnt{%
  This ``certainty effect'' turns into a ``possibility effect'' at the
  other end of the scale.%

  What is supposedly captured by probability weighting is that an
  increase from a 0\% to a 1\% chance of loss significantly taints an
  option, much more than e.g.\ an increase from 20\% to 21\%. At the
  other end of the spectrum, the difference between 99\% and 100\%
  again matters a lot, much more than that between 79\% and 80\%. We
  could make the same predictions by assuming that people have
  preferences about risks, including global features in the utility
  function.
} %
\cmnt{%
  Kahneman notes that buying a lottery ticket also buys ``the right to
  dream pleasantly of winning'', and that by buying insurance you also
  ``eliminate a worry and purchase a peace of mind''
  \cite[318]{kahneman11thinking}, but he nevertheless resists
  broadening the utility function accordingly. %
} %


Prospect theory is clearly an alternative to the simplistic economical
model mentioned above. It is not so obvious whether it is alternative
to the more liberal model we have been studying for most of this
course. Diminishing sensitivity and loss aversion certainly don't
contradict our model. Reference dependence and probability weighting
are a little more subtle.

Our model assumes that if an agent knows the objective probability of
a state, then in decision-making she will weight that state in
proportion to the known probability. Prospect theory says that real
people don't do that. If we measure an agent's credences in terms of
preferences or choices, then the decision weights of prospect theory
are the agent's credences: they play precisely the role of credences
in guiding behaviour. So prospect theory assumes that people
systematically violate the Probability Coordination Principle, since
their credence in low-probability event is greater than the known
objective probability.

Some have argued that the observations that motivate probability
weighting are better explained by redescribing the outcomes and
allowing people to care about things like risk or fairness. However,
there is evidence that sometimes people really do fail to coordinate
their beliefs with known objective probabilities, especially if the
probabilities are communicated verbally. -- Studies show that people's
decision weights are closer to the objective probabilities if they
have experienced these probabilities as relative frequencies in
repeated trials.%
\cmnt{%
  \cite[331]{kahneman11thinking}%
} %
By contrast, when people reason explicitly about probabilities,
systematic mistakes like the base rate fallacy are very common. 

\cmnt{%

Here prospect theory therefore seems to points towards a genuine
respect in which our model doesn't fit actual human behaviour,
although it does not offer a clear picture of how and why people go
wrong when processing verbal information about probabilities.

} %

\cmnt{%
  The mere hypothesis that people's credences deviate from objective
  probabilities in a certain way however arguably doesn't get at the
  heart of the matter. What we'd need is a more general model
  explaining the difference between verbal and non-verbal
  presentation, and the particular mistakes people make.
} %

\cmnt{%
  When you decide not to switch in Monty Hall, your credences are in
  line with your choices, but they violate other norms. Defining an
  alternative model of credence is not easy.
} %

Reference dependence may also raise a genuine challenge. To be sure,
most forms of reference dependence are harmless. Our model can easily
allow for people who care especially about how much they will have in
comparison to what they had before, or in comparison to what their
peers have. But sometimes, the reference point is affected by
intuitively irrelevant features of the context, and that is harder to
square with our model.

\begin{exercise}
  When people compete in sports, average performance sometimes seems
  to function as a reference point insofar as the effort people put in
  to avoid performing below average is higher than the effort they put
  in to exceed the average.%
  \cmnt{%
    \cite[303f.]{kahneman11thinking}.%
  } %
  Can you explain this observation by ``redescribing the outcomes'' in
  the model we have studied, without appealing to reference points?
  $\star$
\end{exercise}

\cmnt{%
  Decision theory can also accommodate for a related fact that even
  counterfactual outcomes matter: if you choose an option $[0.9 ? \$1
  million : \$0]$, then an outcome of $\$0$ will not be regarded as
  neutral; you will be thoroughly disappointed. If there was an
  alternative sure gain, even it was quite small (say, $\$1000$), you
  will probably regret your choice. Avoidance of regret and
  disappointment are psychologically important, but cannot be
  explained by prospect theory or rational choice theory
  \cite[287f.]{kahneman11thinking}. They can easily be accommodated in
  decision theory, since they are simply part of the relevant outcome.
} %

The problematic type of reference dependence is closely related to
so-called \textbf{framing effects}. In experiments, people's choices
can systematically depend on how one and the same decision problem is
described. For example, when presented with a hypothetical situation
in which 1000 people are in danger of death, and a certain act would
save exactly 600 of them, subjects are more favourable towards the act
if it is described in terms of `600 survivors' than if it is described
in terms of `400 deaths'. In prospect theory, the difference might be
explained by a change in reference point: if the outcome is described
in terms of survivors, it is classified as a gain; if it is described
in terms of deaths, it is classified as a loss.

In principle, our liberal model could also explain the relevance of
the description. Perhaps people assign basic value to choosing options
\emph{that have been described in terms of survivors} rather than in
terms of deaths. However, on reflection, most people would certainly
deny that the verbal description of an outcome is of great concern to
them. As in the case of decision weights, a more adequate model would
arguably have to take into account incomplete grasp of a verbally
described scenario. When hearing about survivors, we focus on a
certain attribute of the outcome, on all the people who are
saved. That attribute is desirable. When hearing about deaths, a
different, and much less desirable, attribute of the same outcome
becomes salient.

\cmnt{%
  But is this a failure in decision making, and should we make room
  for it in a descriptively adequate theory of decisions? I'm not
  sure. I suspect it is rather a failure in processing the information
  about the options. Under the positive description (chance of
  success), you really do consider the outcome as more desirable than
  under the negative description. So the problem isn't that you go
  wrong when putting your beliefs and desires into actions. Rather,
  you go wrong when judging the value of a verbally described
  situation. Admittedly, the issue is subtle: your calculation of EUs
  is part of your conforming to decision theory, and isn't this where
  you made the mistake, since you didn't correctly calculate the
  utility of one of the outcomes? But perhaps the more basic reason is
  that you didn't fully understand the content of the outcome you were
  told about. It's not like you were perfectly aware of the relevant
  \emph{proposition}, but sloppy when calculating its utility. The
  presentation-dependence shows that you weren't clear about what
  exactly the outcome would involve: you heard something about success
  and imagined the good case (or the lives saved etc.) and quickly
  gave a value to that, without realizing what else is entailed by the
  presentation. This is less a failure in acting than in
  understanding.
} %

Ideal agents always weigh up all attributes of every possible
outcome. Real agents arguably don't do that, as it requires
considerable cognitive effort.%
\cmnt{%
  One of Gigerenzer's heuristics is to consider only the most
  important attribute of any given outcome.%
} %
As a result, the attributes we consider depend on contextual clues
such as details of a verbal description. Some recent models of
decision making in philosophy and psychology take this kind of
attribute selection into account.

\cmnt{%
  Christian List and Franz Dietrich have a model that involves such
  context-dependent selection of salient features.
} %

\section{Imprecise credence and utility}\label{sec:imprecise}

I'm going to toss three dice. Would you rather get £10 if the sum of
the numbers on the dice is at least 10 or if all three numbers are
different? You'd probably need some time to give a final answer. You
know that the six possible results for every dice have equal
probability, and that the results are independent. But it takes some
effort to infer from that which of the two events I described is more
likely.

A real agent's cognitive system can't explicitly store her credence and
utility for every proposition. It can only store a limited number of
\textbf{constraints} on credences and utilities. A constraint rules
out some credences and utilities, but not others. For example, that
outcomes of die tosses are probabilistically independent is a
constraint; it entails that the probability that three dice will land
six is the product of the probabilities for the individual dice:
$\Cr(\text{Six}_1 \land \text{Six}_2 \land \text{Six}_3) =
\Cr(\text{Six}_1)\Cr(\text{Six}_2)\Cr(\text{Six}_3)$, but it does not
fix what these probabilities are.

So far, we have assumed that taken together, the constraints stored by
an agent's cognitive system are rich enough to determine a unique
credence and utility function. But maybe they don't.  Maybe there are
questions on which you don't have a settled opinion even on
reflection. Or suppose you don't have time for lengthy reflection, or
you're too tired.  In such cases, it would arguably be wrong to model
your attitudes in terms of a single, precise credence function, and a
single, precise utility functions.

Across several disciplines, researchers have developed models that
relax the assumption of precise credences and utilities. The standard
approach is to use \textbf{sets of credence and utility functions}
instead of single functions. The functions in the set are all those
that meet the constraints. Intuitively, each member of the set is a
\emph{refinement} or \emph{precisification} of the agent's
indeterminate state of mind.

The use of sets of credence functions is often motivated by
consideration like the following. How likely do you think it is that
it will snow in Toronto on 7 January 2041? If someone suggested the
probability is 80\%, you might say that's too high; 5\% seems too
low. But 20\% might seem just as plausible to you as 21\%. It would
therefore be wrong to model your state of mind by single and precise
probability. Rather, we should say that your credence is a whole range
of numbers:
\[
\Cr(\emph{Snow}) = [0.1, 0.5].
\]
Here, `$[0.1, 0.5]$' denotes the range of all numbers from 0.1 to 0.5;
it is the range of all numbers your refined credence functions assign
to \emph{Snow}.

\cmnt{%
  Suppose you similarly haven't made up your mind about whether
  Leonardo da Vinci liked porridge: $\Cr(\emph{Porridge}) =
  [0.2,0.7]$. Nonetheless, you might reasonably treat the two
  propositions as independent. (That might be one of the constraints
  encoded in your belief state.) If we model your beliefs in terms of
  credence intervals, we seem to lose the information about
  independence.%
  \cmnt{%
    Recall that \emph{Snow} and \emph{Porridge} are probabilistically
    independent just in case $\Cr(\emph{Snow}/\emph{Porridge}) =
    \Cr(\emph{Snow})$. Plausibly, $\Cr(\emph{Snow} \land
    \emph{Porridge}) = [0.01,0.25]$, so by the ratio formula, we'd
    need to assume that $\frac{[0,01,0.25]}{[0.2,0.7]} = [0.1,0.5]$.%
  } %
} %

You should be skeptical about this line of argument. The term
`probability' in English almost always means objective
probability. When asked about the probability of \emph{Snow}, it is
therefore natural to interpret the question as concerning a certain
objective quantity -- something you could perhaps find out by
developing a sophisticated weather model. But credence, on the
Bayesian conception, is not belief about objective probability. It is
simply strength of belief. You could not find out your credence in
\emph{Snow} by developing a sophisticated weather model.

Nonetheless, there are reasons to extend the Bayesian conception of
credence to allow for sets of credence functions. For example, suppose
we measure (or define) an agent's credences and utilities in terms of
her preferences. Various representation theorems show that if the
agent's preferences satisfy certain axioms, then the preferences are
represented by a \emph{unique} credence and utility function (except
for the conventional choice of zero and unit for utilities). Giving up
uniqueness then means that the agent violates one or more of the
axioms. And it is not implausible that real agents do violate some of
these axioms.

In particular, consider the completeness axiom. Completeness states
that for any propositions $A$ and $B$, the agent either prefers $A$ to
$B$ or $B$ to $A$ or is indifferent between the two. This is trivial
if we define preference in terms of choice. Indeed, presented with a
forced choice between $A$ and $B$, you will inevitably choose either
$A$ or $B$; even indifference can be ruled out. But we've already seen
that if we want to measure credence and utility in terms of
preference, then the relevant preference relation can't be directly
defined in terms of choices. And once we take a step back from choice
behaviour, it seems perfectly possible that you might neither prefer
$A$ to $B$, nor $B$ to $A$, and yet you're not indifferent between the
two. Instead, you haven't fully made up your mind. The two
propositions seem roughly ``on a par'', but you wouldn't say they are
exactly equal in value.

For example, would you rather lose your capacity to hear or your
capacity to walk? You may well have no clear preference, even after
considerable reflection. Does that mean you're exactly indifferent?
Not necessarily. If you were, you should definitely prefer losing the
capacity to hear \emph{and getting £1} to losing the capacity to
walk. In reality, the added £1 probably doesn't make any difference.

\begin{exercise}
  Suppose we define `$\sim$' in terms of `$\succ$', as follows: $A\sim
  B \Leftrightarrow (A \not\succ B) \land (B \not\succ
  A)$. Completeness is then logically guaranteed. But other
  assumptions about preference then fail if you haven't made up your
  mind between certain propositions. For example, it is possible to
  have $A \succsim B$, $B \succsim C$, and $C \succsim A$,
  contradicting transitivity. Explain why (assuming that $A \succsim B
  \Leftrightarrow (A \succ B) \lor (A \sim B)$). $\star\star$
\end{exercise}

\cmnt{%
  Chang thinks parity is different from incomparability. I don't think so.%
} %

\cmnt{%
  (But even transitivity needs to be reconsidered: maybe you've never
  thought about A vs C?)%
} %

So there are reasons to relax completeness, at least if we're
interested in modelling real agents. (Some would say even ideal
agents don't need to have complete preferences.) But we may still
impose an axiom of \textbf{completability}. That is, we can require
that if an agent's preferences violate, say, the Savage axioms because
they fail to rank certain options, then there is a refinement of her
preferences, filling in the missing rankings, that does satisfy the
axioms. Savage's representation theorem then implies that the agent's
preferences are represented by a set of credence and utility
functions.

In sum, even if we don't conflate credence with belief about objective
probability, we might want to model agents as having a set of credence
(and utility) functions. We then have to revise other parts of our
model, to explain how those agents should update their beliefs over
time, and how they should make choices. As it turns out, the required
revisions are not at all straightforward. I will mention just one
problem, concerning rational choice. 

Suppose you have a set of credence and utility functions, because you
haven't made up your mind about certain things, and you face a
choice. According to some of your credence and utility functions, act
$A$ has greatest expected utility; according to others, you should
choose $B$. What should you do? A popular ``permissivist'' answer is
that you are permitted to choose either option.

\begin{exercise}
  Explain how the preference of $A$ to $B$ and $D$ to $C$ in
  Ellsberg's paradox might be justified by the permissivist approach,
  without redescribing the outcomes. (What is the expected utility of
  the four options?)  $\star\star$
\end{exercise}

But now consider the following scenario. You are offered two bets $A$
and $B$, one after the other, on a proposition $H$ about which you
haven't made up your mind. Let's say $\Cr(H) = [0.2,0.8]$. Bet $A$
would give you £1.40 if $H$ and £-1 if $\neg H$. Bet $B$ would give
you £-1 if $H$ and £1.40 if $\neg H$. Assuming for simplicity that
your utility is precise and proportional to the monetary payoff, both
bets have an imprecise expected utility of between -0.52 and
0.92. (For example, the expected utility of the first bet ranges from
$0.2 \times -1 + 0.8 \times 1.40 = 0.92$ to $0.8 \times -1 + 0.2 \times
1.40 = -0.52$.) Your undecided state of mind therefore leaves open
whether accepting either bet is a good idea. On the permissivist
approach, it is permissible for you to refuse both bets. But
arguably, that would be irrational, since the two bets together have a
guaranteed payoff of £0.40. By refusing both bets, you would miss out
on a sure gain.


\cmnt{%
  Perhaps one should choose in some other way then. E.g. by averaging
  the credences. But then there's a threat that we're not taking the
  agent's indecision seriously.
} %
\cmnt{%
\begin{exercise}
  Can't average if independence is a constraint.
\end{exercise}
} %
\cmnt{%

  Another problem concerns learning. According to standard
  Bayesianism, agents learn by conditionalizing on new evidence. How
  does that work if an agent has a set of credence functions? The
  usual answer is that each of the individual credence functions
  should conditionalize.

  Now suppose you don't know the proportion of white balls in an urn
  containing 10 balls. I pick a ball, note its colour, and put it back
  in. What is your credence that I picked a white ball? Arguably, any
  credence distribution is permissible. Now you repeatedly sample a
  ball with replacement and always get a white one. Any credence
  distribution is still permissible! E.g. if one of your initial
  credence functions $\Cr^i_t$ gives 0.9999999999 to there being just
  1 white ball in the urn ($H_1$), and 0.0000000001 to the 10 other
  possibilities (0,2,..,10 balls), then conditionalizing on 10 white
  balls being drawn ($E$) yields
  \begin{align*}
    \Cr^i_{t+1}(H_1) &= \frac{\Cr^i_{t}(E/H_1) \Cr^i_{t}(H_1)}{Cr^i_{t}(E)}\\
    &\approx \frac{0.1^{10} \times 0.9999999999}{0.000000000105}\\
    &\approx 0.95.
  \end{align*}
} %

\section{Further reading}

An accessible overview of some advances in theoretical computer science and their
influence on cognitive science is

\begin{itemize}
\item Samuel Gershman et al: \href{http://gershmanlab.webfactional.com/pubs/GershmanHorvitzTenenbaum15.pdf}{``Computational rationality: A converging paradigm for intelligence in brains, minds, and machines''} (2015)
\end{itemize}

For a brief overview of prospect theory and related models, motivated by
the idea of bounded rationality, see
%
\begin{itemize}
\item Daniel Kahneman: \href{http://www2.econ.iastate.edu/tesfatsi/JudgementAndChoice.MappingBoundedRationality.DKahneman2003.pdf}{``A Perspective on Judgment and Choice''} (2003)
\end{itemize}

The Stanford Encyclopedia article on imprecise probabilities gives a
thorough (and highly opinionated) overview of the models discussed in
section \ref{sec:imprecise}:

\begin{itemize}
\item Saemus Bradley: \href{https://plato.stanford.edu/entries/imprecise-probabilities/}{``Imprecise Probabilities''} (2014)
\end{itemize}

\cmnt{%


Ramsey's approach, without constraints on U and P, seems to allow for
any choice patterns at all.  We can always read off some U and P
relative to which the agent maximizes EU. That shows that these U and
P are not what we intuitively think of or recognize as beliefs and
desires.  For we can easily think of cases where people act
irrationally, against their strongest desires, where they choose worse
options because they didn't consider what they actually knew (think of
chess moves), where they only thought of the most likely outcome, or
the worst outcome, or the best, and so on. All of this would be
conceptually impossible on Ramsey's account.

One might suggest that for this reason we need to impose further,
external rationality constraints on U and P that make satisfying EUT
non-trivial. That's what EUT models in psy and econ often do. Instead
of calling them rationality constraints we could also add constraints
that have nothing to do with normative rationality, but simply seem to
be plausible default assumptions, e.g. no basic desire for saucers of
mud. 

I think this can be a sensible approach, but we have to be
careful. What we wanted to make room for are cases where people
clearly seem to choose options that are non-ideal by the lights of
their beliefs and desires, such as the chess example. Saying that
desires ought to be selfish (for example) or risk-neutral doesn't help
with that, and instead forces us to say that people made irrational
choices in \emph{other} cases where in fact we shouldn't say any such
thing. 

What the above considerations actually suggest is that our concepts of
belief and utility -- not just our pre-theoretic concepts, but also
the concepts that we think should figure in decision theory, deciding
whether an agent was instrumentally rational, as well as the concept
of partial belief that figures in formal epistemology -- that these
concepts are not fully definable in terms of choice dispositions. We
know that the chess player made the wrong move because we know their
goal, and we don't know that simply from their choices. 

Presumably we need a more complicated Bayesian model here that allows
for irrational choices. 

} %



%%% Local Variables: 
%%% mode: latex
%%% TeX-master: "bdrc.tex"
%%% End: