\chapter{Preference}\label{ch:preference}

\cmnt{%

  ``Women loved this impetual Irish adventurer who would rather fight than eat
  and vice-versa.'' (Perelman)
  
  Introduce ``cardinal''.

  I use ``prospect'' for Savage-type conditional events. McClennen uses the word
  for a distribution of objective probabilities over outcomes. Am I misusing the
  term?
  
  On the connection between preference and choice dispositions: One problem here
  arises from indifference. An agent might choose A over B even though she is
  indifferent. Savage 1972:17 points out that this casts doubt on the idea that
  one could define A > B as 'if the agent were required to choose between A and
  B, they would choose A'. One might respond that if an agent is indifferent,
  then it is not the case that /would/ choose A, nor that they would choose B,
  assuming a Lewisian analysis of counterfactuals -- see \cite[20,
  n.5]{ahmed2014evidence}. However, (1) other facts about the world might still
  settle what the agent would do (after all, indifference doesn't imply
  indeterminism), (2) assuming strong centering, the problem still arises for
  pairs of options between which the agent actually made a choice.

  Also, we can't assume that the states and the outcomes are somehow given. We
  might say that the state partition should be any partition which the agent
  believes to be (causally or evidentially) independent of their choice, but
  then we are already assuming information about the agent's beliefs (Ahmed
  p.33f.). Alternatively, we might say that the states should always be
  construed as Lewisian dependency hypotheses (say), but this makes most
  functions from states to acts unintelligible. (compare Gilboa 2009:114ff,
  Joyce 1999:118f., cited in Ahmed 34n16.)

  I might streamline/fix up this chapter by looking at Ahmed's ch.1.
  Specifically:

  - Ahmed says he'll write f > g for 'the agent prefers f to g', $f \geq g$ for
  'the agent does not prefer g to f', and '$f \sim g$ for 'the agent prefers
  neither to the other'.

  - Ahmed lists the axioms in a form that is closer to that of Kreps, but
  provably equivalent to Savage's. He gives a nice motivation for some of them
  (p.20f.).

  - He notes that the inclusion of constant acts in the domain of preference
  effectively extends preferences from acts to outcomes.
  
}%

\section{The ordinalist challenge}

If the utility of an outcome for an agent is not measured by the
amount of money the agent gains or loses, how is it measured? How can
we find out whether an outcome has utility 5 or 500 or -27? What does
it even mean to say that an outcome has utility 5?

At the beginning of the 20th century, doubts arose about the coherence
of numerical utilities. \textbf{Ordinalists} like Vilfredo Pareto
argued that the only secure foundation for utility judgements are
people's choices: if you are given a choice between tea and coffee,
and you choose tea, we can conclude that tea has greater utility for
you than coffee. We may similarly find that you prefer coffee to milk,
etc., but how could we find that your utility for tea is twice your
utility for coffee -- let alone that it has the exact value 5? The
ordinalists concluded that we should give up the conception of utility
as a numerical magnitude.

\cmnt{%
  It is even less clear how any facts about choices could reveal that
  you get more utility from having tea than I get from having coffee.%
} %

Ordinalism posed a serious threat to the idea of expected utility
maximization. If there is no numerical quantity of utility, we can't
demand that rational agents maximize the probability-weighted average
of that quantity, as the MEU Principle requires.

In 1926, Frank Ramsey pointed out that if we look at the choices an
agent makes in a state of uncertainty, we may discover more about the
agent's utility function than how it orders the relevant outcomes --
enough to vindicate the MEU Principle. Ramsey's line of thought was
largely ignored until it was rediscovered by John von Neumann, who
presented a simpler version of it in the 1944 monograph \emph{Game
  Theory and Economic Behaviour}, co-authored with Oskar Morgenstern.
This work is widely taken to provide the foundation of modern expected
utility theory.

Before we have a closer look at von Neumann's idea, let's think a little about
the ordinalist challenge.

Ordinalism was inspired by a wider ``positivist'' movement in science
and philosophy whose aim was to improve scientific reasoning by
discarding reference to seemingly obscure and unobservable facts.
According to positivism, any meaningful statement must have clear
conditions of verification and falsification. If someone puts forward
a hypothesis but can't explain how one could in principle test whether
the hypothesis is true or false, then the hypothesis should be
rejected as meaningless. In psychology, this movement gave rise to
\textbf{behaviourism}, the doctrine that statements about emotions,
intentions, desires, and other psychological states should either be
abandoned or defined in terms of observable behaviour.

Today, behaviourism, and positivism more generally, have been almost
entirely abandoned. One reason for this (to which I already alluded in
the previous chapter) is that people came to appreciate the holistic
character of scientific testing: many statements in successful
scientific theories have observable consequences only in conjunction
with other theoretical assumptions.%
%
\cmnt{%
  Arguably, that is also true for statements about belief and
  desire. A hypothesis about an agent's beliefs alone does not allow
  us to predict what she will do; that also depends on her goals or
  desires.%
} %
More practically, the behaviourist paradigm was found to stand in the
way of scientific progress. It is hard to explain even the behaviour
of simple animals without appealing to inner representational states
like goals or perceptions as causes of the behaviour.

On the basis of these historical developments, it may be tempting to dismiss the
ordinalist challenge as outdated and misguided. But even if their general view
of science was mistaken, the ordinalists raised an important issue.

In chapter \ref{ch:probabilism}, I emphasized that we should not think
of an agent's credences as little numbers written in her head. If your
credence in rain is \nicefrac{1}{2}, then this must be grounded in
other, more basic facts about you -- facts that do not involve the
number \nicefrac{1}{2}. Even if we accept your state of belief as a
genuine internal state, a cause of your behaviour, we need to explain
why we represent that state by the number \nicefrac{1}{2} rather than
\nicefrac{3}{4} or \nicefrac{12}{5}. There's nothing special here
about credence. Numerical representations in scientific models are
always based on non-numerical facts about the represented objects. For
the numerical representations to have meaning, we need to specify what
underlying non-numerical facts the different numbers are meant to
represent.

The same is true for utility. We understand utility to comprise a wide range of
psychological factors, from unconscious aversions to cravings to moral
judgements. What unites all these factors is that they make the relevant
proposition more or less appealing to the agent. The utility of a proposition is
supposed to represents the extent to which the agent, on balance, wants the
proposition to be true -- taking into account all the agent's attitudes towards
the proposition. But it really isn't obvious why a given combination of pro- and
contra-attitudes should be represented by the number 5, say, as opposed to any
other number.

So the ordinalists raised a question that still needs to be answered.
Moreover, there is something to be said for the idea that the answer
should involve the agent's choices.

Consider our practice of attributing conscious or unconscious
motives. A child bullies other children at school. Why does she do
this? One hypothesis is that she simply enjoys the sense of
power. Another is that her aggression is an attempt to hide her
insecurities and protect herself from an unconsciously perceived
threat posed by  other children. A minimal standard for any such
explanation is that the goals it postulates make sense of the child's
behaviour. That is, on the assumption that the child is motivated by
the hypothesized factors, we would expect to see the kind of behaviour
we actually observe.

In general, the main reason to think that an agent has specific goals
or desires is that this would explain her behaviour. The point also
applies to the relative strength of the attributed goals or
desires. We say that my sense of duty was stronger than my desire to
stay in bed because I actually got up. Absent further explanation, the
claim that my desire to stay in bed was stronger, even though
I got up, is unintelligible.

So there is a close connection between an agent's motives or goals or
desires, and her behaviour. What it means to be in a particular
motivational state is, at least in part, to be in a state that
typically leads to particular kinds of behaviour, given suitable
beliefs. If we seek a standard to measure the comparative strength of
different motives, a natural move is to look at their
behavioural consequences.


\cmnt{%
  In current philosophy of mind, this position is known as
  \emph{functionalism}, or \emph{analytical functionalism}, and some
  of its classical defenses include \cite{armstrong68materialist},
  \cite{psychophysical}, and \cite{madpain}. Functionalism is an
  intellectual heir to behaviorism, the view on which an attribution
  of a belief or desire is in some sense equivalent to an attribution
  of complex dispositions to behave. On this view, to say that you
  desire to have eggs for breakfast just is to say that under suitable
  conditions you would choose to have eggs. Apart from the obvious
  difficulty in spelling out ``suitable conditions'', this position
  has problems accounting for beliefs and desires in people who are
  paralysed and may e.g. desire not to be paralysed. Modern-day
  functionalism largely avoids these problems by identifying belief
  and desire not with mere behavioural dispositions, but with real
  psychological states that normally underlie these dispositions.

  Desires, together with beliefs, cause actions. Our Bayesian model
  can be understood as filling in some more details. Utilities sit in
  a network of causal input and output and other mental states.
} %

\cmnt{%

  The most important challenge to functionalism need not concern us
  here. It is the problem of ``phenomenal consciousness'': the fact
  that humans are not merely information-processing devices with
  complex behaviour, but that there is something it is like to be one
  of these devices (see e.g. \cite[ch.1]{chalmers96conscious}). This
  need not concern us here because we are precisely not interested in
  an analysis of \emph{conscious} belief and desire.

} %

\cmnt{%
  Perhaps the first clear statement of modern-day functionalism about
  belief and desire occurs in Frank Ramsey's 1926 paper ``Truth and
  probability'' -- a somewhat obscurely written paper that makes the
  biggest advances in the history of decision theory. We will look at
  this theory in more detail in the next chapter. One of Ramsey's key
  ideas is to measure desire by choices between lotteries.
} %


\section{Scales}

Utility, like credence, mass, or length, is a numerical representation
of an essentially non-numerical phenomenon. All such representations
are to some extent conventional.  We can represent the length of my
pencil as 19 centimetres or as 7.48 inches -- it's the same length
either way. We must take care to distinguish real features of the
represented properties from arbitrary consequences of a particular
representation. For example, it is nonsense to ask whether the length
of my pencil -- the length itself, not the length in any particular
system of representation -- is a whole number. By contrast, it is not
meaningless to ask whether the length of my pencil is greater than the
length of my hand.

The mathematical discipline of measure theory studies the
representation of physical properties by numbers. In the case of
length, as in the case of mass, the conventionality boils down to the
choice of a unit. You can introduce a new measure of length simply by
picking out a particular length and assigning it the number 1. Any
object twice that length will then have a length of 2 in your new
system, and so on. (You can fix the unit by assigning any number
greater than zero to any non-trivial length; it doesn't have to be the
number 1.)

Quantities like mass and length, for which only the unit of
measurement is conventional, are said to have a \textbf{ratio scale},
because even though the particular numbers are conventional, ratios
between them are not: if the length of my arm is four times the length
of my pencil in centimetres, then that is also true in inches,
millimetres, and any other sensible system of measurement.

Temperature is different. Historically, the basis for representing temperature
by numbers was the observation that metals such as mercury expand as the
temperature goes up. Imagine we put some mercury in a narrow glass tube. The
higher the temperature, the more of the glass tube is filled up by the expanding
mercury. To get a numerical measure of temperature, we need to mark two points
on the tube -- for example, by 0 and 100. We can then say that if the mercury
has expanded to $x\%$ of the distance between 0 and 100, then the temperature is
$x$. Anders Celsius suggested to use 0 for the temperature at which water
freezes, and 100 for the temperature at which it boils. Daniel Fahrenheit
instead marked as 0 the lowest temperature he measured in his home town of
Danzig in the winter of 1708/1709, and used 100 for the body temperature of a
healthy human. As a result, 10 degrees Celsius is 50 degrees Fahrenheit, and 20
degrees Celsius is 68 degrees Fahrenheit. The ratio between these two
temperatures is not preserved, so the Celsius scale and the Fahrenheit scale are
not ratio scales.%
\cmnt{%
  By contrast, ratios between \emph{differences} of temperature are not: the
  difference between 10$^\circ$C = xxx and 20$^\circ$C = xxx is twice the
  difference between 5$^\circ$C = xxx and 10$^\circ$C = xxx, in both Celsius and
  Fahrenheit.%
} %
Such scales, where both the zero and the unit are a matter of convention, are
called \textbf{interval scales}.

\cmnt{%
  Yet another, and much simpler, kind of scale is at work when we use
  numbers just to keep track of an object's relative position in an
  order. In some waiting rooms, customers draw tickets with numbers on
  them. A higher number means you're further down in the queue. Apart
  from that, the numbers need not have any significance. The system
  works as long as later customers get higher numbers; the distance
  between the numbers does not matter, nor does it matter whether the
  first customer got the number 1. Here, the only thing that is not a
  matter of convention is the ordering of the numbers: that the second
  customer's number is greater than the first's, and so on. Such
  scales are called \textbf{ordinal scales}.%
} %

The ordinalists held that utility has neither a ratio scale nor an
interval scale, but merely an \textbf{ordinal scale} (hence the name
of the movement). In an ordinal scale, the only thing that is not
conventional is which of two objects is assigned a greater number. For
example, according to the ordinalists, a preference of tea over coffee
and of coffee over milk can be represented by a utility function that
assigns 3 to tea, 2 to coffee, and 1 to milk, but it can also be
represented by a utility function that assigns 300 to tea, 0 to
coffee, and -1 to milk. Either function correctly reflects your
choices.

If the ordinalists were right, we would have to give up the MEU
Principle. Which act in a decision problem maximizes expected utility
would then frequently depend on arbitrary conventions for representing
utility.

By contrast, if utility has an interval scale, then different measures
of utility will never disagree on the ranking of acts in a decision
problem. A ratio scale is not required. 

\begin{exercise}{1}
  In the Mushroom Problem as described by the matrix on page
  \pageref{mushroom-matrix} (section \ref{mushroom-matrix}), not
  eating the mushroom has greater expected utility than eating the
  mushroom. Describe a different assignment of utilities to the four
  outcomes which preserves their ordering but gives eating the mushroom
  greater expected utility than not eating.
\end{exercise}

\cmnt{%
  Exercise: prove that if $U$ is a positive linear transform of $U'$,
  then ratios of differences are preserved. (E.g. Broome 1990 p.75).%
} %

\cmnt{%
  Exercise: show that if $U$ is a positive linear transform of $U'$,
  and money has declining marginal utility in $U$, then it also does
  in $U'$.%
} %

\begin{exercise}{2}
  Suppose two utility functions $U$ and $U'$ differ merely by their
  choice of unit and zero. It follows that there are numbers $x>0$ and
  $y$ such that, for any $A$, $U(A) = x\cdot U'(A) + y$. \cmnt{%
    The number $x$ encodes the ratio of the units, $y$ the difference
    in zeroes. (By comparison, temperature in Fahrenheit equals 1.8
    times temperature in Celsius plus 32.)%
  } %
  Suppose some act $A$ in some decision problem has greater expected
  utility than some act $B$, if the utility of the outcomes is
  measured by $U$. Show that $A$ also has greater expected utility
  than $B$ if the utility of the outcomes is measured by $U'$. (You can
  assume for simplicity that the outcome of either act depends only on
  whether some state $S$ obtains; so the states are $S$ and $\neg
  S$.)%
  \cmnt{%
    The hypothesis that $EU(A) > EU(B)$ then means that
    \[
    U(A)\Cr(S_1) + U(A)\Cr(S_2) > U(B)\Cr(S_1) + U(B)\Cr(S_2).
    \]
    Adding $y$ to both sides and multiplying by $x$, with $x>0$, we get:
    \[
    x[U(A)\Cr(S_1)+ U(A)\Cr(S_2)]+y > x[U(B)\Cr(S_1) + U(B)\Cr(S_2)]+y.
    \]
    With a little algebra, this entails
    \[
    [xU(A)+y] \Cr(S_1)+ [xU(A)+y]\Cr(S_2) > [xU(B)+y]\Cr(S_1) + [xU(B)+y]\Cr(S_2)].
    \]
    Which is to say that $EU'(A) > EU'(B)$. 
  } %
\end{exercise}
  
So if we want to rescue the MEU Principle from ordinalist skepticism,
we don't need to explain what makes it the case that your utility for
tea is 5 rather than 500; we can accept that the precise numbers are a
matter of conventional representation. Nor do we need to explain what
makes your utility for tea twice your utility for coffee; such ratios
also needn't track anything real. But we do have to explain what makes
it the case that once we arbitrarily set your utility for tea as 5 and
your utility for coffee as 0, then your utility for milk is fixed at,
say, -7.

\cmnt{%
  Is an agent's value function more like temperature or more like
  mass?  We could say that there's a privileged zero: complete
  indifference. But how does indifference show up in an agent's
  preferences? Also, if a world has zero, and zero means indifference,
  and indifference means status quo, then conditionalization changes
  values. That's not what we want.
} %

\cmnt{%
  Before the ordinalist revolution, it was commonly assumed that
  utility has a ratio scale, with the zero fixed by the point where an
  agent feels neither pleasure nor pain. But it is not clear how the
  zero point could be determined through an agent's choices. In any
  case, it turns out that to make sense of the MEU Principle, an
  interval scale is enough. So the standard approach today is to
  assume that utility, like temperature, is relative to a zero and a
  unit.
} %


\section{Utility from preference}

I am now going to describe John von Neumann's proposal for how to
determine an agent's utility function from her choice dispositions.

Recall from the previous chapter that an agent's utility function is
fixed by the utility assigned to the propositions I called
\emph{concerns}. Intuitively, a concern settles everything that
matters to the agent, leaving open only questions towards which the
agent is indifferent. To make the following discussion a little more
concrete (and to bypass some problems that will occupy us later),
let's imagine an agent who is only interested in getting certain
``rewards'', which may be lumps of money or commodity bundles or
pleasant sensations. I will use lower-case letters $a,b,c,\ldots$ for
rewards. Our goal is to find the agent's numerical utility for
$a,b,c,\ldots$.

We will determine the agent's utilities from her \textbf{preferences},
which we assume to represent her choice dispositions. For example,
let's say the agent would choose reward $a$ if she were given a choice
between $a$ and $b$. She then prefers $a$ to $b$. The ordinalists did
not challenge the idea that people have preferences in this sense.

Let's introduce some shorthand notation:
%
\begin{align*}
  a \succ b &\Leftrightarrow \text{The agent prefers $a$ to $b$}.\\
  a \sim b &\Leftrightarrow \text{The agent is indifferent between $a$ and $b$.}\\
  a \succsim b & \Leftrightarrow \text{The agent prefers $a$ to $b$ or is indifferent between them.}
\end{align*}
%
(Note that `$\succ$', `$\sim$', and `$\succsim$' had a different
meaning in section \ref{sec:comparative-credence}. You always have to
look at the context to figure out what these symbols mean.)

We saw that to defend the MEU Principle, we can choose an arbitrary
unit and zero for the utility scale. So let's take arbitrary rewards
$a$ and $b$ such that $b \succ a$ and set $U(a) = 0$ and $U(b) = 1$.
This corresponds to Celsius choosing zero as the temperature at which
water freezes and 100 as the temperature at which water boils.

\begin{exercise}{2}
  If the agent is indifferent between all rewards, then our procedure
  stalls at this step. Nonetheless, we can easily find a utility
  function for such an agent. What does it look like? 
\end{exercise}

Having fixed the utility of two rewards $a$ and $b$, here is how we
can determine the utility of any other reward $c$. We distinguish
three cases, depending on how the agent ranks $c$ relative to $a$ and
$b$.

Suppose first that $c$ ``lies between'' $a$ and $b$ in the sense that
$b \succ c$ and $c \succ a$. To find the utility of $c$, we look at
the agent's preferences between $c$ and a \textbf{lottery} between $a$
and $b$. By a `lottery between $a$ and $b$', I mean an event that
leads to $a$ with some objective probability $x$ and otherwise to $b$.
For example, suppose we offer our agent a choice between $c$ for sure
and the following gamble $L$: we'll toss a fair coin; on heads the
agent gets $a$, on tails $b$. By the Probability Coordination
Principle, the expected utility of the gamble is%
\cmnt{%
\[
  EU(L) = \nicefrac{1}{2} \cdot U(a) + \nicefrac{1}{2} \cdot U(b) =  
   \nicefrac{1}{2} \cdot 0 + \nicefrac{1}{2} \cdot 1 = \nicefrac{1}{2}. 
\]
} %
\nicefrac{1}{2}. So if the agent is indifferent between the lottery and $c$, and
the agent obeys the MEU Principle, we can infer that the agent's utility for $c$
is \nicefrac{1}{2}.

\begin{exercise}{1}
  Suppose an agent's utility is 0 for $a$, 1 for $b$, and
  \nicefrac{1}{2} for $c$. Draw a decision matrix representing a
  choice between $c$ and $L$, and verify that both options have
  expected utility \nicefrac{1}{2}.
\end{exercise}

\begin{exercise}{2}
  Why do we need to assume that the agent obeys the Probability
  Coordination Principle?
\end{exercise}

If the agent isn't indifferent between $L$ and $c$, we try other
lotteries, until we find one for which the agent is indifferent
between the lottery and $c$.%
\cmnt{%
  For example, if the agent prefers $L$ to $c$, we can infer that
  $U(c) < \nicefrac{1}{2}$.%
} %
For example, suppose the agent is indifferent between $c$ and a gamble
$L'$ where she would get $a$ with probability \nicefrac{4}{5} and $b$
with probability \nicefrac{1}{5}. Since the expected utility of this
gamble is \nicefrac{1}{5}, we could infer that the agent's
utility for $c$ is \nicefrac{1}{5}.

We have assumed that $c$ lies between $a$ and $b$. What if the agent
prefers $c$ to both $a$ and $b$? Then we will look for a lottery between
$a$ and $c$ for which the agent is indifferent between $b$ and the
lottery. For example, if the agent is indifferent between $b$ for sure
and a gamble $L''$ where she gets either $a$ or $c$ with equal
probability, then $c$ must have utility 2. That's because the expected
utility of $L''$ is
\[
  EU(L'') = \nicefrac{1}{2} \cdot U(a) + \nicefrac{1}{2} \cdot U(c) =  
  0 + \nicefrac{1}{2} \cdot U(c) = \nicefrac{1}{2} \cdot U(c). 
\]
Since the agent is indifferent between $L''$ and $b$, which has a
guaranteed utility of 1, the gamble must have expected utility 1. So
$1 = \nicefrac{1}{2} \cdot U(c)$. And so $U(c) = 2$. In general, if
the agent is indifferent between $b$ and a lottery that leads to $c$
with probability $x$ and $a$ with probability $1-x$, then
$U(c) = \nicefrac{1}{x}$.

\cmnt{%
  $EU(L'') = x U(a) + (1-x)U(c) = (1-x)U(c)$. If this is 1, then $U(c)
  = 1/(1-x)$.%
} %

\begin{exercise}{2}
  Can you complete the argument for the case where the agent prefers
  both $a$ and $b$ to $c$?
\end{exercise}

In this manner, we can determine the agent's utility for all rewards from her
preferences between rewards and lotteries. And so we can answer the ordinalist
challenge: we can define numerical utilities as given by any utility function we
could read off in this way from an agent's preferences. This is the official
definition of `utility' in many economics textbooks. (A simpler definition of a
merely ordinal concept of utility in terms of preferences is also frequently
used for applications where uncertainty can be ignored.)

\section{The von Neumann and Morgenstern axioms}\label{sec:vnm}

The method described in the previous section assumes that the agent
obeys the MEU Principle. At first glance, that may seem strange. The
ordinalists argued that the MEU Principle made no sense; how can we
respond to them by \emph{assuming} the principle? Besides, doesn't
application of the MEU Principle presuppose that we already know the
agent's utilities?

The trick is that we are applying the principle backwards. Normally,
when we apply the MEU Principle, we start with an agent's beliefs and
desires and try to find out her (optimal) choices. Now we start with her
choices and try to find out her desires, relying on the Probability
Coordination Principle to fix the relevant beliefs.

There is nothing dodgy about this. Whenever we want to measure a
quantity whose value can't be directly observed, we have to rely on
assumptions about how the quantity relates to other things that we can
observe. Together with the Probability Coordination Probability, the
MEU Principle tells us what lotteries an agent should be disposed to
accept if she has a given utility function. If she wouldn't accept
those lotteries, we can infer that she doesn't have the utility
function, and so we look at other lotteries until we find a match.

You may wonder, though, what happened to the normativity of the MEU
Principle. If we follow von Neumann's method to define an agent's
utility function, won't the agent automatically come out as obeying
the MEU Principle?

Not quite. It's true that \emph{if the method works}, then the agent will
evaluate certain lotteries by their expected utility, relative to the utility
function identified by the method. But the method is not guaranteed to work. Nor
does it settle how the agent evaluates arbitrary lotteries or decision problems
in which the objective probabilities are unknown.

To illustrate the first point, we have assumed that if an agent ranks
some reward $c$ in between $a$ and $b$, then the agent is indifferent
between $c$ and some lottery between $a$ and $b$. That is not a
logical truth. An agent could in principle prefer $c$ to any lottery
between $a$ and $b$, yet still prefer $c$ to $a$ and $b$ to $c$. Von
Neumann's method does not identify a utility function for such an
agent.

Von Neumann and Morgenstern investigated just what conditions an
agent's preferences must satisfy in order for the method to work, and
for it to guarantee that the agent evaluates arbitrary lotteries by
their expected utility. To state these conditions, I will assume that
`$\succ$', `$\sim$', and `$\succsim$' are defined not just for basic
rewards but also for lotteries between rewards as well as ``compound
lotteries'' whose payoff is another lottery. For example, if I toss a
fair coin and offer you lottery $L$ on heads and $L'$ on tails, that
would be a compound lottery.

Here are the conditions we need. `$A$', `$B$', `$C$' range over
arbitrary lotteries or rewards.

\begin{genericthm}{Completeness}
  For any $A$ and $B$, exactly one of $A \succ B$, $B\succ A$, or $A
  \sim B$ is the case.
\end{genericthm}
\cmnt{%
  Completeness is plausibly implied by an interpretation of preference
  in terms of choice dispositions.

  Incomparability: If there is only one pro-attitude, any apparent
  incomparability is a matter of indeterminacy. One might postulate
  several pro-attitude, e.g. selfish vs moral, short-term vs
  long-term, to account for more substantial incomparability. But (a)
  when it comes to actions, some sort of comparison is inescapable,
  and (b) arguably whatever motivates these distinctions also
  motivates even more fine-grained distinctions e.g. between a
  career-related pro-attitude and a hobby-related attitude etc., which
  is getting silly. In any case, (c) we're only developing a model.)

} %
\vspace{-2mm}
\begin{genericthm}{Transitivity}
  For any $A$, $B$, and $C$, if $A \succsim B$ and $B\succsim C$ then
  $A \succsim C$.
\end{genericthm}
\cmnt{%
  Money pumps?

  I should add an exercise on transitivity, something like Broome's
  Maurice. First ask about the rationalization: options individuated
  more narrowly. Follow-up: Can't you be money-pumped here? Isn't that
  still a sign of irrationality? In practice, you might be
  money-pumped once, if you didn't see it coming.

} %
\vspace{-2mm}
\begin{genericthm}{Continuity}
  For any $A$, $B$, and $C$, if $A \succ B$ and $B \succ C$ then
  there are lotteries $L_1$ and $L_2$ between $A$ and $C$ such
  that $A \succ L_1 \succ B$ and $B \succ L_2 \succ C$.
\end{genericthm}
\vspace{-2mm}
\begin{genericthm}{Independence (of Irrelevant Alternatives)}
  For any $A$, $B$, and $C$, if $A \succ B$, and $L_1$ is a lottery
  that leads to $A$ with some probability $x$ and otherwise to $C$,
  and $L_2$ is a lottery that leads to $B$ with probability $x$ and
  otherwise to $C$, then $L_1 \succ L_2$.
\end{genericthm}
\vspace{-2mm}
\begin{genericthm}{Reduction (of Compound Lotteries)}
  If a $L_1$ and $L_2$ are two (possibly compound) lotteries that lead
  to the same rewards with the same objective probabilities, then $L_1
  \sim L_2$.
\end{genericthm}

Von Neumann and Morgenstern proved that if (and only if) an agent's
preferences satisfy all these conditions, then there is a utility
function $U$, determined by the method from the previous section,
which \emph{represents} the agent's preferences in the sense that
$A \succ B$ just in case $U(A) > U(B)$, and $A \sim B$ just in case
$U(A) = U(B)$. Moreover, the function $U$ is unique except for the
choice of unit and zero. (That is, any two functions $U$ and $U'$ that
represent the agents preferences differ at most in their choice of
unit and zero.) This result is known as the \textbf{von
  Neumann-Morgenstern Representation Theorem}.

\cmnt{%
  \begin{genericthm}{The von Neumann and Morgenstern Representation
      Theorem}
    If (and only if) an agent's preferences satisfy Completeness,
    Transitivity, Independence, Continuity, and Reduction, then there
    is a utility function $U$ such that
    \begin{enumerate}
    \item[(a)] $A \succ B$ just in case $U(A) > U(B)$, and
    \item[(b)] the utility of a lottery is the sum of the probability
      of each outcome times the utility of that outcome.
    \end{enumerate}
    Moreover, any two functions $U$ and $U'$ that satisfy (a) and (b)
    differ only by their choice of unit and zero.
  \end{genericthm}
} %

So if we follow von Neumann and Morgenstern's definition of utility, then the
MEU Principle for choices involving lotteries will automatically be satisfied by
any agent whose preferences satisfy the above conditions -- Completeness,
Transitivity, etc. The normative claim that an agent ought to evaluate lotteries
by their expected utility therefore reduces to the claim that their preferences
ought to satisfy the conditions. For this reason, the conditions are also known
as the \textbf{axioms of expected utility theory}.

So von Neumann and Morgenstern offered not only a response to the
ordinalist challenge. They  also offered a powerful argument for the MEU Principle. The
argument could be spelled out as follows.

\begin{enumerate}
  \itemsep0em
\item The preferences of a rational agent satisfy Completeness,
  Transitivity, Continuity, Independence, and Reduction.
\item If an agent's preferences satisfy these conditions, then (by the
  Representation Theorem) they are represented by a utility function
  $U$ relative to which the agent ranks options by their expected utility.
\item That utility function $U$ is the agent's true utility function.
\item Therefore: a rational agent ranks options by their expected utility.
\end{enumerate}

\cmnt{%
  To assess the plausibility of the MEU Principle, we should therefore
  have a closer look at the axioms.
} %

\begin{exercise}{1}
  (An example due to John Broome.) Maurice would choose to go to Rome
  if he were offered a choice between Rome and going to the mountains,
  because the mountains frighten him. Offered a choice between staying
  at home and going to Rome, he would prefer to stay at home, because
  he finds sightseeing boring. But if he were offered a choice between
  mountaineering and staying at home, he would choose the mountains
  because it would be cowardly, he believes, to stay at home. Which of
  the axioms does Maurice appear to violate?
\end{exercise}


\section{Utility and credence from preference}

In chapter \ref{ch:probabilism} we looked at the betting
interpretation, which attempts to derive an agent's credences from her
choice dispositions. We saw that the approach relied on implausible
assumptions about the agent's utility function. Meanwhile, we
have learned from von Neumann and Morgenstern how we might derive an
agent's utility function from her choice dispositions. Ramsey (in
1926) argued that the two tasks can be combined. He showed how we
might simultaneously determine both an agent's credences and her
utilities from her choices.%
%
\cmnt{%
It is worth studying this approach not just because it offers a
dramatic improvement over the betting interpretation of credence, but
also because it promises to fill a gap in von Neumann and
Morgenstern's argument for the MEU Principle. At best, their argument
showed that agents should rank lotteries by their expected
utility. But not all choices involve lotteries. In a lottery, the
agent knows the objective probability of each outcome. In real life,
we often face choices in which the objective probabilities (if they
even exist) are unknown. Yet the MEU Principle is supposed to apply in
these cases, too.%
} %
%
Ramsey's idea was rediscovered and streamlined by Leonard Savage in
his \emph{Foundations of Statistics} (1954) -- the second most
influential book in the history of decision theory, after \emph{Game
  Theory and Economic Behaviour}. I will briefly describe Savage's
main result.

Like von Neumann and Morgenstern, Savage begins with some conditions
(``axioms'') on an agent's comparative preference relation, which we
assume to reflect the agent's choice dispositions. This time, the
preference relation is defined over a set of basic rewards as well as
\emph{conditional prospects}. A conditional prospect is an event that
leads to some reward $a$ if some state of the world $X$ obtains and
otherwise to a possibly different reward $b$. I will abbreviate such a
prospect by `$\bet{X}{a}{b}$' (pronounced `if $X$ then $a$ else $b$').
We also allow for conditional prospects in which the outcomes are not
basic rewards but further conditional prospects. The intuitive thought
is that any act in any decision problem corresponds to a conditional
prospect. In the mushroom problem from chapter \ref{ch:overview}, for
example, eating the mushroom amounts to choosing the prospect
$\bet{\text{Poisonous}}{\text{Dead}}{\text{Satisfied}}$; not eating
the mushroom amounts to choosing
$\bet{\text{Poisonous}}{\text{Hungry}}{\text{Hungry}}$.

Savage's axioms are a little more complicated than those of von
Neumann and Morgenstern. As before, we need Completeness and
Transitivity -- I won't repeat them. We also need to assume that the
agent is not indifferent between all rewards, otherwise would have no
means of discovering her beliefs:
%
\begin{genericthm}{Non-Triviality}
  There are rewards $a$ and $b$ for which $a\succ b$.
\end{genericthm}

The Independence axiom is redefined as follows, to reflect the change
from lotteries to conditional prospects.
%
\cmnt{%
  I follow Joyce 1999.
} %
%
\begin{genericthm}{Independence (of Irrelevant Alternatives)} 
  If two prospects $A$ and $B$ lead to different outcomes only under
  condition $X$, then for any $C$, $A \succ B$ iff $\bet{X}{A}{C}
  \succ \bet{X}{B}{C}$.
\end{genericthm}

Instead of Continuity and Reduction we have the following
conditions. (Don't worry if you can't make immediate sense of them.)
%
\begin{genericthm}{Nullity} 
  If $A \pref B$ and $\bet{X}{A}{C} \not\pref \bet{X}{B}{C}$, then
  for all $A',B'$, $\bet{X}{A'}{C} \not\pref \bet{X}{B'}{C}$.
\end{genericthm}
\cmnt{%
  Note that if $A \pref B$ and $\bet{X}{A}{C} \not\pref
  \bet{X}{B}{Z}$, then the probability of $X$ is 0.%
}%
\vspace{-2mm}
\begin{genericthm}{Stochastic Dominance} 
  If $A \pref B$ and $A'\pref B'$, then $\bet{X}{A}{B} \pref
  \bet{Y}{A}{B}$ iff $\bet{X}{A'}{B'} \pref \bet{Y}{A'}{B'}$.
\end{genericthm}
\cmnt{%
  Note that if $\bet{A}{X}{Y} \pref \bet{B}{X}{Y}$ then the
  probability of $A$ is greater than that of $B$.%
}%
\vspace{-2mm}
\begin{genericthm}{State Richness} 
  If $A \pref B$, then for all $C$ there is a finite number of
  mutually exclusive and jointly exhaustive states $X_1,\ldots,X_n$
  such that for all $X_i$, $\bet{X_i}{C}{A} \pref B$ and $A \pref
  \bet{X_i}{C}{B}$.
\end{genericthm}
\cmnt{%
  The idea is that you can partition the states into finitely many
  cells each of which has a probability so small that it doesn't
  affect the ordering between $X$ and $Y$ if each of those is replaced
  by $Z$ in one cell of the partition.
}%
\vspace{-2mm}
\begin{genericthm}{Averaging} 
  It is not the case that for all outcomes $O$ that might result from
  prospect $A$ under condition $X$, $\bet{X}{A}{B} \succ
  \bet{X}{O}{B}$, nor is it the case that $\bet{X}{O}{B} \succ
  \bet{X}{A}{B}$ for all such outcomes.
\end{genericthm}

\cmnt{%
  \begin{genericthm}{Savage's Representation Theorem}
    If an agent's preferences satisfy Completeness, Transitivity,
    Independence, Nullity, Stochastic Dominance, State Richness, and
    Averaging, then there is a (bounded) utility function $U$ and a
    probability function $\Cr$ such that $A \pref B$ iff the expected
    utility of $A$, relative to $\Cr$ and $U$, is greater than that of
    $B$. Moreover, $\Cr$ is unique and $U$ is unique up to the choice of
    zero and unit.
  \end{genericthm}
} %
  
\textbf{Savage's Representation Theorem} states that if (but not only
if) an agent's preferences satisfy all these conditions, then there is
a utility function $U$ and a probability function $\Cr$ such that $A
\succ B$ iff the expected utility of $A$, relative to $\Cr$ and $U$,
is greater than that of $B$. Moreover, $\Cr$ is unique and $U$ is
unique expect for the choice of zero and unit.

What can this do for us? Well, suppose we \emph{define} an agent's credence
and utility functions as the functions $\Cr$ and $U$ whose existence
and uniqueness (modulo conventional choice of zero and unit, for $U$)
is guaranteed by Savage's theorem, provided the agent's preferences
satisfy the axioms. If we accept the axioms as genuine norms of
rationality, we can then explain what non-numerical facts the
credences and utilities of rational agents are meant to represent:
they represent certain patterns in the agent's preferences and
therefore ultimately in her choice dispositions. For on the present
approach, saying that a rational agent has credences $\Cr$ and
utilities $U$ is equivalent (by the proposed definition of $\Cr$ and
$U$) to a certain claim about the agent's preferences. We no longer
need the betting interpretation, and we have answered the ordinalist challenge.

In addition, the present approach promises a more comprehensive
argument for the MEU Principle than the argument we got from von
Neumann and Morgenstern. Their argument only showed that agents should
rank \emph{lotteries} by their expected utility. But not all choices
involve lotteries. In real life, agents often face conditional
prospects in which they don't know the objective probability of the
various outcomes. Why should they rank such prospects by their
expected utility? Savage's answer is that if they don't, then they
don't satisfy his axioms.

We also get a new argument for probabilism -- the claim that rational
degrees of belief satisfy the probability axioms. Again, the
requirement reduces to the preference axioms: on the proposed
definition of credence, any agent who obeys these axioms automatically
has probabilistic credences. If you don't have probabilistic
credences, you violate the axioms.

% this goes by too fast

\begin{exercise}{2}
  Can you spell out the argument for probabilism I just outlined in
  more detail, in parallel to the argument for the MEU Principle I
  spelled out at the end of section \ref{sec:vnm}?
\end{exercise}


\cmnt{%
  Now we get an argument for both the MEUP and probabilism.  The
  theorem shows that if you violate probabilism and maximise EU, then
  you fail one of the axioms. In other words, if you're going to act
  in accordance with your beliefs, and you want to satisfy the axioms,
  you must be probabilistic. And if you want to be probabilistic, you
  should MEU.
} %

It should now be clear why the results of Savage and von Neumann and
Morgenstern are widely taken to provide the foundations of expected
utility theory. The results not only seem to show how an agent's
credence and utility function can be measured in terms of overt
choices, they also suggest that our main norms -- probabilism and the
MEU Principle -- reduce to certain conditions on choices. To finish
the job, it seems, we only have to convince ourselves that these
conditions (the ``axioms'') are genuine norms of rationality.

In fact, doubts have been raised about every single one of the axioms.
We will turn to some of these worries in chapter \ref{ch:risk}. In the
meantime, I want to flag a different kind of problem.

\section{Preference from choice?}\label{sec:preferences-choices}

Von Neumann and Savage take as their starting point an agent's
preferences, represented by the relations $\succ$, $\sim$, and
$\succsim$. Informally, we have interpreted `$A \succ B$' as stating
that the agent would choose $A$ if she were offered a choice between
$A$ and $B$. The representation theorems are then supposed to explain
how an agent's utilities (or utilities and credences) can be measured
by her choice dispositions.

An agent's \emph{dispositions} reflect what she \emph{would} do if
such-and-such circumstances were to arise. It should be clear that we
can't just look at the agent's actual choice behaviour, since most
agents are not confronted with all the choices from which von Neumann
or Savage would derive their utility function.

\begin{exercise}{2}
  Suppose we define `$A \sim B$' as `the agent has faced a choice between $A$
  and $B$ and expressed indifference', and `$A \succ B$' as `the agent has faced
  a choice between $A$ and $B$ and expressed a preference for $A$. Which of the
  von Neumann and Morgenstern axioms then become highly implausible (no matter
  what exactly we mean by ``expressing'' indifference or preference)?
\end{exercise}

But now one of the problems for the betting interpretation, from
section \ref{sec:problem-betting}, returns with a vengeance. If an
agent is in fact not facing a choice between two options $A$ and $B$,
then offering her the choice would change her beliefs. Among other
things, she would come to believe that she faces that
choice. According to the MEU Principle, the agent in the hypothetical
choice situation should choose whichever option maximizes expected
utility relative to the beliefs and desires she has in that
situation. So if we interpret preferences in terms of hypothetical
choices, then we cannot assume that a rational agent prefers $A$ to
$B$ just in case $A$ has greater expected utility than $B$ relative to
the agent's actual beliefs and desires.

\cmnt{%
  If we interpret preferences that way, some of the axioms will
  fail. E.g., you may well ``prefer'' a lottery between a and b over
  both a and b. %
} %

The problem gets worse if we drop the simplifying assumptions that
agents only care about lumps of money, commodity bundles, or pleasant
sensations. Suppose one thing you desire (one ``reward'') is peace in
Syria, another is being able to play the piano. Von Neumann's
definition then determines your utilities in part by your preferences
between peace in Syria and a lottery that leads to peace in Syria with
objective probability \nicefrac{1}{4} and to an ability to play the
piano with probability \nicefrac{3}{4}. Savage's method will similarly
look at your preferences between peace in Syria and various prospects
like \bet{\emph{Rain}}{\emph{Peace}}{\emph{Piano}} -- an imaginary act
that leads to peace in Syria if it rains and to an ability to play the
piano if it doesn't rain. If you thought you'd face this bizarre
choice, your beliefs would surely be quite different from your actual
beliefs.
% Indeed, no matter what you pick in the hypothetical choice
% situation, it is guaranteed that there is either peace in Syria or you
% can play the piano. So you could only believe that you face the
% hypothetical choice if you are sure one of these propositions is true.
% Clearly we can't assume that your credences in the hypothetical choice

Even in the rare case where an agent actually faces one of the
relevant choices, we arguably can't infer that whichever option she
chooses has greater expected utility for her. 

For one thing, people can have false beliefs about their options. If
your real choice is between water and wine, but you think it is
between petrol and wine (because you think the water is petrol), we
can't infer from your choice of wine that you prefer wine to water.
Von Neumann and Savage presuppose that agents are never mistaken about
their options.

\cmnt{%
  Also, what if you're not aware of a certain option?%
} %

What's more, if an agent chooses $A$ over an alternative $B$, we can't infer that
she genuinely prefers $A$. Perhaps she is indifferent and chooses $A$ merely
because she has to make a choice. Arguably, choice dispositions can't tell apart
$A \succ B$ and $A \sim B$.

\cmnt{%
\begin{exercise}
  Yet another problem: Suppose an agent picks $A$ in a choice between
  $A$ and $B$. Assuming the agent is aware that her options are indeed
  $A$ and $B$, why can we still not infer that she prefers $A$ to $B$?
  (Hint: Which of the following possibilities are compatible with the
  agent's choice -- $A \succ B$, $B \succ A$, $A \sim B$?) $\star$
\end{exercise}
} %

\cmnt{%
  All this shows that the preference relations that figure in the
  axioms and theorem can not straightforwardly be interpreted in terms
  of observable choices or choice dispositions.

  That's a problem for behaviourism/functionalism, not so much for the
  normative applications?%
} %

The upshot of all these problems is that we need to distinguish (at
least) two notions of preference. One represents the agent's choice
dispositions: whether she would choose $A$ over $B$ in a hypothetical
situation in which she faces that choice. The other represents the
agent's current ranking of hypothetical prospects or lotteries:
whether by the lights of her current beliefs and desires, $A$ is
better than $B$. Von Neumann and Savage at best demonstrated how to
derive utilities and credences from preferences in the second sense.

This could still be valuable. For example, we might still get
interesting arguments for probabilism and the MEU Principle. Moreover,
there is plausibly \emph{some} connection between preference in the
second sense and choices dispositions, so even though we haven't fully
solved the measurement problem for credences and utilities, one might
hope that we are at least a few steps closer.

\cmnt{%

We will encounter even more problems for von Neumann and Morgenstern
and Savage in chapter \ref{ch:risk}. You may already get a glimpse of
them if you consider how the model of utility from the previous
chapter meshes with the accounts of von Neumann and Morgenstern and
Savage. In the previous chapter, we saw how an agent's utility
function is determined by her credence function and her value function
-- the utility function restricted to maximally specific
possibilities. In a sense the individual worlds are the basic objects
of desire. But in the framework von Neumann and Morgenstern or Savage,
the ``rewards'' can't be individual possible worlds. Can you see why?

} %

\cmnt{%

There are serious problems hiding in the concepts of a lottery or
prospect and a reward. The problems do not arise if the agent in
question only cares about her present level of pleasure or wealth. But
suppose we want to be more liberal and allow for other
values. For example, suppose an 


Von Neumann and Morgenstern assume that any ``reward'' can either come
about directly or as the result of a lottery. But that is not
obvious. If the reward is that there are no lotteries then this cannot
come about as the result of a lottery. The approach also assumes that
the way the lottery is determined is not itself an object of value. If
I desire all coins to land heads, then two fully specified rewards may
specify that all coins land heads. And then it makes no sense to have
a lottery that yields $R_1$ on heads and $R_2$ on tails. 

Both of these problems also arise in Savage's framework. If
conditional prospects $\bet{X}{A}{B}$ are understood as hypothetical
acts, then an agent may well care about which of these acts take
place. (For example, suppose her moral or religious beliefs imply that
certain such acts are wrong.) So two ``rewards'' $A$ and $B$ might
both entail the absence of $\bet{X}{A}{B}$. And then the conditional
prospect is logically impossible. Moreover, suppose the agent cares
about $X$. Then a fully specific reward will entail whether or not $X$
obtains. So there can't be a 


\cmnt{%

  In the previous chapter, we saw how an agent's utility function is
  determined by her credence function and her value function -- the
  utility function restricted to maximally specific possibilities. In
  a sense the individual worlds are the basic objects of desire. But
  can we take them to be the rewards? Arguably not. For consider a
  prospect $\bet{A}{w}{w'}$. If $A$ is true in $w$ and not $w'$, the
  bet is equivalent to $w$. Moreover if the act $\bet{A}{w}{w'}$ takes
  place neither in $w$ nor in $w'$, the prospect is logically
  impossible.

} %


Jeffrey/Bolker and Joyce prove RTs that solve the second
problem. However, weakening the assumptions about the space of
prospects makes it no longer possible to derive a unique credence
function.

In a way, that should not be too surprising or concerning. Why should
an agent's credences be entirely determined by her preferences? If you
want to know what someone believes, looking at their behaviour is
certainly useful. But so is looking at what they see. 


\cmnt{%
  From the previous section we already know how to compute utilities
  from an agent's basic value function. Since her basic value function
  encodes everything she cares about, it is clearly not a single
  introspectible quantity, nor a conscious judgement (people can be
  wrong). So what does it mean if we say that an agent assigns value
  0.3 or -7 or $3\,000$ to a possible world?
} %

} %

\section{Further reading}

For some recent discussion of the idea that utility and credence are
derived from preferences, see (for example)

\begin{itemize}
\item Samir Okasha: \href{https://research-information.bristol.ac.uk/files/70387986/On_Interpretation_Decision_TheoryECONPHILrevised.pdf}{``On the Interpretation of Decision Theory''} (2016),
\item Daniel Hausman: ``Mistakes about Preferences in the Social
  Sciences'' (2011).
\end{itemize}
You may notice that Okasha and Hausman mean different things by
`preference'.

A useful survey of many further representation theorems in the style
of those we have reviewed is
\begin{itemize}
\item Peter Fishburn: ``Utility and Subjective Probability'' (1994).
\end{itemize}

\begin{essay}
  Do you think an agent's choice dispositions can in principle reveal
  all her goals and values? If yes, can you explain how? If no, can
  you explain why not?
\end{essay}


% According to behavioural inter- pretations of preference, roughly, what it
% means to prefer option a to option b is just that one actually does, or
% hypothetically would choose option a rather than option b. For a widely
% discussed recent defense, see Faruk Gul and Wolfgang Pesendorfer. The case for
% mindless eonomics. In Andrew Caplin and Andrew Schotter, editors, The
% Foundations of Positive and Normative Economics. Oxford University
% Press, 2008.


% That U is defined from < is sometimes called "constructivism". See Dreier
% (1996) and Velleman (1993/2000) for defences of constructivism about utility




%%% Local Variables: 
%%% mode: latex
%%% TeX-master: "bdrc.tex"
%%% End:
