\chapter{Modelling Rational Agents}\label{ch:overview}

\section{Overview}

\cmnt{%
  Dan Ross: ``There is a dense and intricate web of connections associated with
  ‘rationality’ in the Western cultural tradition, and the word has
  often been used to normatively marginalize characteristics as normal
  and important as emotion, femininity and empathy''.
  
  ``In this article, ‘economic rationality’ will be used in the
  technical sense shared within game theory, microeconomics and formal
  decision theory, as follows. An economically rational player is one
  who can (i) assess outcomes, in the sense of rank-ordering them with
  respect to their contributions to her welfare; (ii) calculate paths
  to outcomes, in the sense of recognizing which sequences of actions
  are probabilistically associated with which outcomes; and (iii)
  select actions from sets of alternatives (which we'll describe as
  ‘choosing’ actions) that yield her most-preferred outcomes, given
  the actions of the other players. We might summarize the intuition
  behind all this as follows: an entity is usefully modeled as an
  economically rational agent to the extent that it has alternatives,
  and chooses from amongst these in a way that is motivated, at least
  more often than not, by what seems best for its purposes. (For
  readers who are antecedently familiar with the work of the
  philosopher Daniel Dennett, we could equate the idea of an
  economically rational agent with the kind of entity Dennett
  characterizes as intentional, and then say that we can usefully
  predict an economically rational agent's behavior from ‘the
  intentional stance’.)''
} %

In this course, we will study a general model of belief, desire, and
rational choice. At the heart of the model lies a certain conception
of how beliefs and desires combine to produce actions.

Let's start with an example.
%
\begin{example}[The Miner Problem]\label{ex:miners}
  Ten miners are trapped in a shaft and threatened by rising water.
  You don't know whether the miners are in shaft $A$ or in shaft
  $B$. You can block the water from entering one shaft, but you can't
  block both. If you block the correct shaft, all ten will survive.
  If you block the wrong shaft, all of them will die. If you do
  nothing, one miner (the shortest of the ten) will die.
\end{example}

What should you do?

There's a sense in which the answer depends on the location of the
miners. If the miners are in shaft $A$, it's best to block shaft $A$;
if they are in $B$, you should block $B$. The problem is that you need
to make your choice without knowing where the miners are. You can't
let your choice be guided by the unknown location of the miners. The
question on which we will focus is therefore not what you should do
\emph{in light of all the facts}, but what you should do \emph{in
  light of your information}. In other words, we want to know what a
rational agent would do in your state of uncertainty.

A similar ambiguity arises for goals or values. Arguably, it is better
to let one person die than to take a risk of ten people dying. But the
matter isn't trivial, and many philosophers would disagree. Suppose
you are one of these philosophers: you think it would be wrong to
sacrifice the shortest miner. \emph{By your values}, it would be
better to block either shaft $A$ or shaft $B$.

When we ask what an agent should do in a given decision problem, we
will always mean what they should do in light of whatever they believe
about their situation and of whatever goals or values they happen to
have. We will also ask whether those beliefs and goals are themselves
reasonable. But it is best to treat these as separate questions.

So we have three questions:
\begin{enumerate}
  \itemsep0em 
\item How should you act so as to further your goals in the light of
  your beliefs?
\item What should you believe?
\item What should you desire? What are rational goals or values?
\end{enumerate}
These are big questions. By the end of this course, we will not have
found complete and definite answers, but we will at least have
clarified the questions and made some progress towards an answer.

To begin, let me introduce a standard format for thinking about
decision problems.

\begin{exercise}
  A doctor recommends small pox vaccination for an infant, knowing
  that around 1 in 1 million children dies from the vaccination. The
  infant gets the vaccination and dies. Was the doctor's
  recommendation wrong? Or was it wrong in one sense and right in
  another? If so, can you explain these senses?
\end{exercise}


\section{Decision matrices}\label{sec:decision-matrices}

In decision theory, decision problems are traditionally decomposed
into three ingredients, called `acts', `states', and `outcomes'.

The \textbf{acts} are the options between which the agent has to
choose. In the Miner Problem, there are three acts: \emph{blocking
  shaft $A$}, \emph{blocking shaft $B$}, and \emph{doing nothing}.
(`Possible act' would be a better name: if, say, you decide to do
nothing, then blocking shaft $A$ is not an actual act; it's not
something you do, but it's something you could have done.)

The \textbf{outcomes} are whatever might come about as a result of the
agent's choice. In the Miner Problem, there are three relevant
outcomes: all miners survive, all miners die, and all but one
survive. (Again, only one of these will actually come about, the
others are merely possible outcomes.)

Each of the three acts leads to one of the outcomes. But you don't
know how the outcomes are associated with the acts. For example, you
don't know whether blocking shaft $A$ would lead to all miners
surviving or to all miners dying. It depends on where the miners are.

This dependency between acts and outcomes is captured by the states. A
\textbf{state} is a possible circumstance on which the result of the
agent's choice depends. In the Miner Problem, there are two relevant
states: that the miners are in shaft $A$, and that the miners are in
shaft $B$. (In real decision problems, there are often many more
states, just as there are many more acts.)

We can now summarize the Miner Problem in a table, called a
\textbf{decision matrix}:
%
\begin{center}
  \begin{tabular}{|r|c|c|}\hline
    \gr & \gr Miners in $A$ & \gr Miners in $B$\\\hline
    \gr Block shaft $A$ & all 10 live & all 10 die \\\hline
    \gr Block shaft $B$ & all 10 die & all 10 live \\\hline
    \gr Do nothing & 1 dies & 1 dies \\\hline
  \end{tabular}
\end{center}

The rows in a decision matrix always represent the acts, the columns
the states, and the cells the outcome of performing the relevant act
in the relevant state.

Let's do another example.
\begin{example}[The mushroom problem]
  You find a mushroom. You're not sure whether it's a delicious
  \emph{paddy straw} or a poisonous \emph{death cap}. You wonder
  whether you should eat it.
\end{example}

Here the decision matrix might look as follows. Make sure you
understand how to read the matrix.

\begin{center}
  \begin{tabular}{|r|c|c|}\hline
    \gr & \gr Paddy straw & \gr Death cap\\\hline
    \gr Eat & satisfied & dead \\\hline
    \gr Don't eat & hungry & hungry\\\hline
  \end{tabular}
\end{center}

Sometimes the ``states'' are actions of other people, as in the next
example.

\begin{example}[The Prisoner Dilemma]\label{ex:pd}
  You and your partner have been arrested for some crime and are
  separately interrogated. If you both confess, you will both serve
  five years in prison. If one of you confesses and the other remains
  silent, the one who confesses is set free, the other has to serve
  eight years. If you both remain silent, you can only be convicted of
  obstruction of justice and will both serve one year.
\end{example}

The Prisoner Dilemma combines two decision problems: one for you and
one for your partner. We could also think about a third problem which
you face as a group. Let's focus on the decision you have to
make. Your choice is between confessing and remaining silent. These
are the acts. What are the possible outcomes? If you only care about
your own prison term, the outcomes are 5 years, 8 years, 0 years, and
1 year. Which act leads to which outcome depends on whether your
partner confesses or remains silent. These are the states. In matrix
form:

\begin{center}
  \begin{tabular}{|r|c|c|}\hline
    \gr & \gr Partner confesses & \gr Partner silent\\\hline
    \gr Confess & 5 years & 0 years \\\hline
    \gr Remain silent & 8 years & 1 year \\\hline
  \end{tabular}
\end{center}

Notice that if your goal is to minimize your prison term, then
confessing leads to the better outcome no matter what your partner
does. So that is what you should do.

I've assumed you only care about your own prison term. What if you
also care about the fate of your partner? Then your decision problem
is not adequately summarized by the above matrix, as the cells in the
matrix don't say what happens to your partner.

The outcomes in a decision problem must always include everything that
matters to the agent. So if you care about your partner's sentence,
the matrix should look as follows.

\begin{center}
  \begin{tabular}{|r|c|c|}\hline
    \gr & \gr Partner confesses & \gr Partner silent\\\hline
    \gr Confess & you 5 years, partner 5 years & you 0 years, partner 8 years \\\hline
    \gr Remain silent & you 8 years, partner 0  years & you 1 year, partner 1 years\\\hline
  \end{tabular}
\end{center}
%
Now confessing is no longer the obviously best choice. For example, if
your goal is to minimize the combined prison term for you and your
partner, then remaining silent is better no matter what your partner
does.

\begin{exercise}\label{e:rock}
  Draw the decision matrix for the game \emph{Rock, Paper, Scissors},
  assuming all you care about is whether you win.
\end{exercise}

\cmnt{%
\begin{exercise}
  In the \emph{Ultimatum Game}, two players are offered £100 on the
  condition that they can agree how to divide the money. Player 1
  begins by proposing a split; for example: ``£80 to me, £20 to
  you''. Player 2 can then accept or reject the proposal. If she
  rejects, both get nothing. 

  Draw the decision matrix for player 2's choice, assuming player 1
  suggested ``£80 to me, £20 to you'' and player 2 only cares about
  how much money she gets.
\end{exercise}

\begin{exercise}
  When people play the Ultimatum Game, unfair offers like ``£80 to me,
  £20 to you'' are often rejected by the second player. What does this
  tell us  about the second player's values? Can you draw a
  decision matrix reflecting these values?
\end{exercise}
} %


\section{Belief, desire, and degrees}

To solve a decision problem we need to know the agent's goals and
beliefs. Moreover, it is usually not enough just to know \emph{what}
the agent believes and desires; we also need to know \emph{how strong}
these attitudes are.

Let's return to the mushroom problem. Suppose you like eating a
delicious mushroom, and you dislike being hungry and being dead. We
might therefore label the outcomes `good' or `bad', reflecting your
desires:

\begin{center}
  \begin{tabular}{|r|c|c|}\hline
    \gr & \gr Paddy straw & \gr Death cap\\\hline
    \gr Eat & satisfied (good) & dead (bad) \\\hline
    \gr Don't eat & hungry (bad) & hungry (bad) \\\hline
  \end{tabular}
\end{center}
Now it looks like eating the mushroom is the better option: not
eating is guaranteed to lead to a bad outcome, while eating at least
gives you a shot at a good outcome.

The problem is that you probably prefer being hungry to being
dead. Both outcomes are bad, but one is much worse than the other. So
we need to represent not only the \textbf{valence} of your desires --
whether an outcome is something you'd like or dislike -- but also
their \textbf{strength}. 

An obvious way to represent both valence and strength is to label the
outcomes with numbers, like so:

\begin{center}
  \begin{tabular}{|r|c|c|}\hline
    \gr & \gr Paddy straw & \gr Death cap\\\hline
    \gr Eat & satisfied (+1)  & dead (-100) \\\hline
    \gr Don't eat & hungry (-1) & hungry (-1) \\\hline
  \end{tabular}
\end{center}
The outcome of eating a paddy straw gets a value of +1, because it's
moderately desirable. The other outcomes are negative, but death (-100)
is rated much worse than hunger (-1).

The numerical values assigned to outcomes are called
\textbf{utilities} (or sometimes \textbf{desirabilities}). Utilities
measure the relative strength and valence of desire. We will have a
lot more to say on what that means in due course.

We also need to represent the strength of your beliefs. Whether you
should eat the mushroom arguably depends on how confident you are that
it is a Paddy straw.

Here again we will represent the valence and strength of beliefs by
numbers, but this time we'll only use numbers between 0 and 1. If the
agent is certain that a given state obtains, then her degree of belief
is 1; if she is certain that the state does \emph{not} obtain, her
degree of belief is 0; if she is completely undecided, her degree of
belief is 1/2. These numbers are called \textbf{credences}.

In classical decision theory, we are not interested in the agent's
beliefs about the acts or the outcomes, but only in her beliefs about
the states. The fully labelled mushroom matrix might therefore look as
follows, assuming you are fairly confident, but by no means certain,
that the mushroom is a paddy straw.
%
\label{mushroom-matrix}
\begin{center}
  \begin{tabular}{|r|c|c|}\hline
    \gr & \gr Paddy straw (0.8) & \gr Death cap (0.2)\\\hline
    \gr Eat & satisfied (+1)  & dead (-100) \\\hline
    \gr Don't eat & hungry (-1) & hungry (-1) \\\hline
  \end{tabular}
\end{center}
%
The numbers $0.8$ and $0.2$ in the column headings specify your degree
of belief in the two states.

The idea that beliefs vary in strength has proved fruitful not just in
decision theory, but also in epistemology, philosophy of science,
artificial intelligence, statistics, and other areas. The keyword to
look out for is \textbf{Bayesian}: if a theory or framework is called
Bayesian, this usually means it involves degrees of belief. The name
refers to the Thomas Bayes (1701--1761), who made an important
contribution to the movement. We will look at some applications of
``Bayesianism'' in later sessions.

Much of the power of Bayesian models derives from the assumption that
rational degrees of belief satisfy the mathematical conditions on a
probability function. Among other things, this means that the
credences assigned to the states in a decision problem must add up to
1. For example, if you are 80 percent (0.8) confident that the
mushroom is a paddy straw, then you can't be more than 20 percent
confident that the mushroom is a paddy straw. It would be OK to
reserve some credence for further possibilities, so that the credence
in the paddy straw possibility and the death cap possibility add up to
less than 1. But then our decision matrix should include further
columns for the other possibilities.

So rational degrees of belief have a certain formal structure.  What
about degrees of desire? At first glance, these don't seem have much
of a structure. For example, the fact that your utility for eating a
paddy straw is +1 does not seem to entail anything about your utility
for eating a death cap. Nonetheless, we will see that utilities also
have a rich formal structure -- a structure that is entangled with the
structure of belief.

We will also discuss more substantive, non-formal constraints on
belief and desire. Economists often assume that rational agents are
self-interested, and so the term `utility' is often associated with
personal wealth or welfare. That's not how we will use the term. Real
people don't just care about themselves, and there is nothing wrong
with that.

\begin{exercise}
  Add utilities and (reasonable) credences to your decision matrix for
  \emph{Rock, Paper, Scissors}.
\end{exercise}

\section{Solving decision problems}\label{sec:solving}

Suppose we have drawn up a decision matrix and filled in the credences
and utilities. We then have all we need to solve the decision problem
-- to say what the agent should do in light of her goals and beliefs.

Sometimes the task is easy because some act is best in every
state. We've already seen an example in the Prisoner Dilemma, given
that all you care about is minimizing your own prison term. The fully
labelled matrix might look like this:

\begin{center}
  \begin{tabular}{|r|c|c|}\hline
    \gr & \gr Partner confesses (0.5) & \gr Partner silent (0.5)\\\hline
    \gr Confess & 5 years (-5)& 0 years (0)\\\hline
    \gr Remain silent & 8 years (-9)& 1 year (-1) \\\hline
  \end{tabular}
\end{center}

In the lingo of decision theory, confessing \textbf{dominates}
remaining silent. In general, an act $A$ dominates an act $B$ if $A$
leads to an outcome with greater utility than $B$ in every possible
state. An act is \textbf{dominant} if it dominates all other acts. If
there's a dominant act, it is always the best choice (by the light of
the agent).

The Prisoner Dilemma is famous because it refutes the idea that good
things will always come about if people only look after their own
interests. If both parties in the Prisoner Dilemma only care about
themselves, they end up 5 years in prison. If they had cared enough
about each other, they could have gotten away with 1.

Often there is no dominant act. Recall the mushroom problem.
\begin{center}
  \begin{tabular}{|r|c|c|}\hline
    \gr & \gr Paddy straw (0.8) & \gr Death cap (0.2)\\\hline
    \gr Eat & satisfied (+1)  & dead (-100) \\\hline
    \gr Don't eat & hungry (-1) & hungry (-1) \\\hline
  \end{tabular}
\end{center}
It is better to eat the mushroom if it's a paddy straw, but better not
to eat it if it's a death cap. So neither option is dominant.

You might say that it's best not to eat the mushroom because eating
could lead to a really bad outcome, with utility -100, while not
eating at worst leads to an outcome with utility -1. This is an
instance of \textbf{worst-case reasoning}. The technical term is
\textbf{maximin} because worst-case reasoning tells you to choose the
option that \emph{max}imizes the \emph{min}imal utility.

People sometimes appeal to worst-case reasoning when giving health
advice or policy recommendations, and it works out OK in the mushroom
problem. Nonetheless, as a general decision rule, worst-case reasoning
is indefensible. 

Imagine you have 100 sheep who have consumed water from a contaminated
well and will die unless they're given an antidote. Statistically, one
in a thousand sheep die even when given the antidote. According to
worst-case reasoning there is consequently no point of giving your
sheep the antidote: either way, the worst possible outcome is that all
the sheep will die. In fact, if we take into account the cost of the
antidote, then worst-case reasoning suggests you should not give the
antidote (even if it is cheap).

Worst-case reasoning is indefensible because it doesn't take into
account the likelihood of the worst case, and because it ignores what
might happen if the worst case doesn't come about. A sensible decision
rule should look at all possible outcomes, paying special attention to
really bad and really good ones, but also taking into account their
likelihood.

The standard recipe for solving decision problems therefore evaluates
each act by the \emph{weighted average} of the utility of all possible
outcomes, weighted by the likelihood of the relevant state, as given
by the agent's credence.

Let's first recall how simple averages are computed. If we have $n$
numbers $x_1, x_2, \ldots, x_n$, then the average of the numbers is
\[
\frac{x_1 + x_2 + \ldots + x_n}{n} = \tfrac{1}{n} \cdot x_1 + \tfrac{1}{n} \cdot x_2 + \ldots + \tfrac{1}{n} \cdot x_n.
\]
Here each number is given the same weight, $\nicefrac{1}{n}$. In a
weighted average, the weights can be different for different numbers.

Concretely, to compute the weighted average of the utility that might
result  from eating the mushroom, we multiply the utility of each
possible outcome (+1 and -100) by your credence in the corresponding
state, and then add up these products. The result is called the
\textbf{expected utility} of eating the mushroom.
\[
EU(\text{Eat}) = 0.8 \cdot (+1) + 0.2 \cdot (-100) = -19.2.
\]

In general, if there are $n$ states $S_1,\ldots, S_n$, your credence
in these states is $c_1,\ldots,c_n$ respectively, and the outcome of
act $A$ in these states has utility $u_1,\ldots,u_n$ respectively,
then the expected utility of $A$ is defined as
\[
EU(A) = c_1 \cdot u_1 + \ldots + c_n \cdot u_n.
\]
You'll often see this abbreviated using the `sum' symbol $\sum$:
\[
EU(A) = \sum_{i=1}^n c_i \cdot u_i.
\]
It means the same thing.

\cmnt{%
  Better use format from ch.9: So here is the general rule for
  determining expected utilities in the new way. If there are $n$
  states $S_1,\ldots, S_n$, and a given act $A$ leads to outcomes
  $O_1,\ldots,O_n$ respectively in these states, then
  \[
  EU_e(A) = \Cr(S_1/A) \cdot U(O_1) + \ldots + \Cr(S_n/A) \cdot U(O_n).
  \]
} %

\cmnt{%
  xxx Mention more general notion: EU is an instance of a useful
  general notion: expectation.  ``We can then calculate your
  expectation for the quantity. While there are subtleties we will
  return to later, the basic idea of an expectation is to multiply
  each value the quantity might take by your credence that it’ll take
  that value, then add up the results.''
} %


Note that the expected utility of eating the mushroom is -19.2 even
though the most likely outcome has positive utility. A really bad
outcome can seriously push down an act's expected utility even if the
outcome is quite unlikely. 

Let's calculate the expected utility of not eating the mushroom:
\[
EU(\text{Not Eat}) = 0.8 \cdot -1 + 0.2 \cdot -1 = -1.
\]
No surprise here. If all the numbers $u_1,\ldots,u_n$ are the same,
their weighted average will again be that number.

Now we can state one of the central assumptions of our model:

\begin{genericthm}{The MEU Principle}
  Rational agents maximize expected utility. 
\end{genericthm}
%
That is, when faced with a decision problem, rational agents choose an
option with greatest expected utility.

\cmnt{%
  Rationality is a complex, multi-faceted, and somewhat fuzzy
  notion. We will not bother defining or analysing it, for we will in
  any case only be interested in certain aspects of rationality. One
  aspect of rationality is consistency of belief: if a rational agent
  is certain that the cat is on the sofa, she will not also be certain
  that the cat is on a tree. There are similar norms on goals and
  desires. Of special importance to us are norms connecting beliefs
  and goals on the one hand with behaviour on the other.
} %

\begin{exercise}
  Assign utilities to the outcomes in the Prisoner Dilemma, assign
  credences to the states, and compute the expected utility of the two
  acts.
\end{exercise}

\begin{exercise}
  Assign utilities to the outcomes in the Miner Problem, assign
  credences to the states, and compute the expected utility of the
  three acts.
\end{exercise}

\begin{exercise}
  Explain why the following decision rule is not generally reasonable:
  \emph{Identity the most likely state; then choose an act which
    maximizes utility in that state.}
\end{exercise}

\begin{exercise}
  Show that if there is a dominant act, then it maximizes
  expected utility.
\end{exercise}

\cmnt{%
  \begin{exercise}
    What about this weaker claim: ``If one believes that an act will not
    bring about the best outcome, one should not choose it.'' Show that
    that's also inadequate!
  \end{exercise}
} %

\begin{exercise}\label{e:exam}
  When applying dominance reasoning or the MEU Principle, it is
  important that the decision matrix is set up correctly. 

  A student wants to pass an exam and wonders whether she ought to
  study. She draws up the following matrix.
  \begin{center}
    \begin{tabular}{|r|c|c|}\hline
      \gr & \gr Will Pass (0.5) & \gr Won't Pass (0.5) \\\hline
      \gr Study & Pass \& No Fun (1) & Fail \& No Fun (-8) \\\hline
      \gr Don't Study & Pass \& Fun (5) & Fail \& Fun (-2) \\\hline
    \end{tabular}
  \end{center}
  She finds that not studying is the dominant option. 

  The student has correctly identified the acts and the outcomes in
  her decision problem, but the states are wrong. In an adequate
  decision matrix, the states must be independent of the acts: whether
  a given state obtains should not be affected by which act the
  student chooses. 

  Can you draw an adequate decision matrix for the student's decision
  problem?
\end{exercise}

\begin{exercise}[Pascal's Wager]\label{e:pascal}
  The first recorded use of the MEU Principle outside gambling dates
  back to 1653, when Blaise Pascal presented the following argument
  for leading a pious life. (I paraphrase.)

  \emph{An impious life is more pleasant and convenient than a
    pious life. But if God exists, then a pious life is rewarded by
    salvation while an impious life is punished by eternal
    damnation. Thus it is rational to lead a pious life even if one
    gives quite low credence to the existence of God.}
  
  Draw the matrix for the decision problem as Pascal conceives it and
  verify that a pious life has greater expected utility than an
  impious life.
\end{exercise}

\begin{exercise}
  Has Pascal identified the acts, states, and outcomes correctly? If
  not, what did he get wrong?
\end{exercise}

\cmnt{%
\begin{exercise}
  Your casino offers customers a game where they can roll a fair die
  and win the resulting number, in pounds. How much should you charge
  for each round of the game so that you can expect to break even in
  the long run?
\end{exercise}
} %


\section{The nature of belief and desire}

A major obstacle to the systematic study of belief and desire is the
apparent familiarity of the objects. We all know what beliefs and
desires are; we have been thinking and talking about them from an
early age and continue to do so almost every day. We may sometimes ask
how a peculiar belief or unusual desire came about, but the nature and
existence of the states themselves seems unproblematic. It takes some
effort to appreciate what philosophers call \textbf{the problem of
  intentionality}: the problem of explaining \emph{what makes it the
  case} that an agent has certain beliefs and desires. 

For example, some people believe that there is life on other planets,
others don't. What accounts for this difference? Presumably the
difference between the two kinds of people can be traced to some
difference in their brains, but what is that difference, and how does
a certain wiring and chemical activity between nerve cells constitute
a belief in alien life? More vividly, what would you have to do in
order to create an artificial agent with a belief in alien life?
(Notice that producing the sounds `there is life on other planets' is
neither necessary nor sufficient.)

If we allow for degrees of belief and desire (as we should), the
problem of intentionality takes on a slightly different form: what
makes it the case that an agent has a belief or desire \emph{with a
  given degree}? For example, what makes it the case that my credence
in the existence of alien life is greater than 0.5? What makes it the
case that I give greater utility to sleeping in bed than to sleeping
on the floor?

These may sound like obscure philosophical questions, but they turn
out to be crucial for a proper assessment of the models we will study.

I already mentioned that economists often identify utility with
personal wealth or welfare. On that interpretation, the MEU
Principle says that rational agents are guided solely by the expected
amount of personal wealth or welfare associated with various
outcomes. Yet most of us would readily sacrifice some amount of wealth
or welfare in order to save a child drowning in a pond. Are we
thereby violating the MEU Principle? 

In general, we can't assess the MEU Principle unless we have some idea
of how utility and credence (and thereby expected utility) are to be
understood. There is a lot of cross-talk in the literature because
authors tacitly interpret these terms in slightly different ways.

So to put flesh on the MEU Principle, we will have to say more about
what we mean by `credence' and `utility'. I have informally introduced
credence as degree of belief, and utility as degree of desire, but we
should not assume that the mental vocabulary we use in everyday life
precisely carves our objects of study at their joints. 

For example, the word `desire' sometimes suggests an unreflective
propensity or aversion. In that sense, rational agents often act
against their desires, as when I refrain from eating a fourth slice of
cake, knowing that I will feel sick afterwards. By contrast, an
agent's utilities comprise everything that matters to the agent --
everything that motivates them, from bodily cravings to moral
principles. It does not matter whether we would ordinarily call these
things `desires'.

\cmnt{%
  Here the familiarity of belief and desire presents another
  obstacle. I have introduced credence as degree of belief, and
  utility as degree of desire; thus it is tempting to assume that
  ordinary English judgements about (degrees of) belief and desire
  reliably track an agent's credences and utilities. But that
  temptation should be resisted. There is no good reason to think that
  the mental vocabulary we use in everyday life carves our objects of
  study at their joints. On the contrary.

  Consider desire. In ordinary thought and talk, the objects of desire
  are often presented as particular things: I desire a slice of
  cake. On reflection, however, it is clear that these desires don't
  pertain directly to the relevant things, but to a certain relation
  between the subject and the things: I desire to \emph{eat} the slice
  of cake. We will therefore depart from ordinary language and always
  take the objects of utility to be complete states of affairs.

  Here's another famous puzzle about desire reports in ordinary
  language. Imagine Hans believes that there is a ghost in his
  attic. Suppose I say, `Hans desires that the ghost in his attic will
  be quiet tonight'. Observe that any state of affairs in which the
  ghost in Hans's attic is quiet is also a state of affairs in which
  there is a ghost in Hans's attic. So my statement (`Hans
  desires\ldots') seems to attribute to Hans a desire in the third of
  the following three possibilities:
  \begin{enumerate}
  \item[$A$] There is no ghost in Hans's attic.
  \item[$B$] There is a noisy ghost in Hans's attic.
  \item[$C$] There is a quiet ghost in Hans's attic.
  \end{enumerate} 
  Yet for some complicated reason linguists are still debating over,
  `Hans desires that the ghost in his attic will be quiet' only
  conveys that Hans prefers state $C$ to state $B$, and not that he
  prefers $C$ to $A$. The statement could be true even if Hans has a
  strong desire to have no ghost in the attic -- no noisy ghost but
  also no quiet ghost. So we must be careful not to be misled by
  ordinary statements about desire.

  In addition, return to the drowning child. If you jump into the pond
  realizing an obligation to save the child, do you \emph{desire} to
  save the child? Ordinary judgement is ambivalent. It is not a
  contradiction to say that rational agents sometimes do things which
  they don't desire to do. Here `desire' seems be used in a sense on
  which, roughly speaking, to desire an act means to feel an ``emotion
  of aversion or propensity'', as Hume put it, when imagining the act.
  If utility measures strength of desire in this sense, the MEU
  Principle once again becomes highly implausible. We shall therefore
  understand utility in another way.
} %

\cmnt{%
  Loosely speaking, the ``desire'' part will capture how an agent
  wants the world to be; the ``belief'' part how it represents the
  world to be. Desires will include a diversity of things that one
  wouldn't normally call desires: fears, commitments, etc. The belief
  part will also include things one wouldn't call beliefs and it
  doesn't include things one would attribute as beliefs (as in
  Pierre's case, or indexical cases).
} %

The situation we face is ubiquitous in science. Scientific theories
often involve expressions that are given a special, technical
sense. Newton's laws of motion, for example, speak of `mass' and
`force'. But Newton did not use these words in their ordinary sense;
nor did he explicitly give them a new meaning: he nowhere defines
`mass' and `force'. Instead, he tells us what these things \emph{do}:
objects accelerate at a rate equal to the ratio between the force
acting upon them and their mass, and so on. These laws implicitly
define the Newtonian concept of mass and force.

We will assume a similar perspective on credence and utility. That is,
we won't pretend that we have a perfect grip on these quantities from
the outset. Instead, we'll start with a vague and intuitive conception
of credence and utility and then successively refine this conception
as we develop our model.

\cmnt{%

Papineau: In general, the greater degree of belief an agent attaches
to some proposition p, the more that agent will be inclined to perform
actions that will bring good results if p.

The resulting concept of beliefs and desire has a pragmatic element,
insofar as it takes actions to be true touchstone of whether someone
has a given attitude, as opposed to e.g.\ feelings of conviction or
verbal statements. It is easy to pay lip service to something. People
say they're certain that p, but if you ask them to bet they refuse.

There is also lots of psychological research indicating that we are
not very good at verbally reporting our attitudes and the reasons for
our actions (see e.g. \cite{wilson08unseen}). So if we want to use a
decision theoretic model for explaining actions, or for judging
whether an action was reasonable given the subject's attitudes, it
might be better not to rely on what the subject reports.

} %

One last point. I emphasize that we are studying a \textbf{model} of
belief, desire, and rational choice. Outside fundamental physics,
models always involve simplifications and idealisations. In that
sense, ``all models are wrong'', as the statistician George Box once
put it. The aim of scientific models (outside fundamental physics) is
not to provide a complete and fully accurate description of certain
events in the world -- the diffusion of gases, the evolution of
species, the relationship between interest rates and inflation -- but
to isolate simple and robust patterns in these events. It is not an
objection to a model if it leaves out details or fails to explain
various edge cases.

The model we will study is an extreme case insofar as it abstracts
away from most of the contingencies that make human behaviour
interesting. Our topic is not specifically human behaviour and human
cognition, but what unifies all types of rational behaviour and
cognition.

\section{Further reading}

The use of decision matrices, dominance reasoning, and the MEU
Principle are best studied through examples. A good starting point is
the Stanford Encyclopedia entry on Pascal's Wager, which carefully
dissects exercise \ref{e:pascal}:

\begin{itemize}
\item Alan H\'ajek: \href{http://plato.stanford.edu/entries/pascal-wager/}{Pascal's Wager} (2017)
\end{itemize}

Some general rules for how to identify the right acts, states, and
outcomes can be found in

\begin{itemize}
\item James Joyce: ``Decision Problems'', chapter 2 of \emph{The
    Foundations of Causal Decision Theory} (1999)
\end{itemize}

We will have a lot more to say about credence, utility, and the MEU
Principle in later sessions. You may find it useful to read up on
modelling in general and on the functionalist conception of beliefs
and desires. Some recommendations:

\begin{itemize}
  \itemsep0em 
  \cmnt{%
  \item Robert Stalnaker: ``The Problem of Intentionality'', chapter 1
    of \emph{Inquiry} (1984)
  } %

\item Alisa Bokulich: \href{http://www.romanfrigg.org/Links/MS2/Bokulich.pdf}{``How scientific models can explain''} (2011)

\item Ansgar Beckermann:
  \href{https://www.uni-bielefeld.de/(en)/philosophie/personen/beckermann/pbint_ve.pdf}{``Is
    there a problem about intentionality?''} (1996)

\item Mark Colyvan: \href{http://www.colyvan.com/papers/iinm.pdf}{``Idealisations in normative models''} (2013)

  \cmnt{%
  \item Nancy Cartwright: ``Models: Parables v Fables'' (2010)
  } %
  
\end{itemize}

\begin{essay}
  Rational agents proportion their beliefs to their evidence. Evidence
  is what an agent learns through perception. So could we just as well
  explain rational choice on the basis of an agent's
  \emph{perceptions} and desires rather than her \emph{beliefs} and
  desires?
\end{essay}

\cmnt{%
\begin{essay}
  In ``The Problem of Intentionality'', Robert Stalnaker first defends
  and then rejects what he calls the \emph{pragmatic analysis} of
  belief and desire. What is the analysis? Explain in your own words
  why Stalnaker rejects it. Can you think of a response?
\end{essay}
} %

\cmnt{%

  \textbf{Lecture}:

  \begin{itemize}

  \item Clarify mechanics: lectures + tutorials; must attend lectures;
    before coming to lectures, work through notes, hand in exercises
    (with student numbers!); grades (20\% for empty answers, essay);
    website.

  \item Lecture: won't read out, but tell me what you didn't
    understand.

  \item Intro: course will provide a glimpse at a powerful and
    influential model of rationality; three elements: belief, desire,
    action [picture]. We'll talk about internal structure of each, and
    their connections.

  \item (1.1) Subjective ``right''. Three questions.

  \item (1.2) Decision matrices.

  \item (1.3) Desirability of outcomes; credence in states.

  \item (1.4) Solving DPs: find act with MEU.

  \item (1.5) Nature of B and D.

  \end{itemize}

  \textbf{Tutorial}:

  \begin{itemize}

  \item Ultimatum game: play in pairs; tabulate offers and acceptances; discuss: why offer? DP? discuss: why accept/reject? DP money/values.

  \item Pascal's wager: draw silly DP (act `belief', infinite utility, two states; mention problem of underspec. acts: pious only for mercenary reasons, and further acts, e.g. mixing)

  \item Student problem

  \end{itemize}

} %

%%% Local Variables: 
%%% mode: latex
%%% TeX-master: "bdrc.tex"
%%% End: